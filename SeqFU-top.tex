\documentclass{article}
\usepackage{texmacros}
\usepackage{titlesec}
\usepackage[margin=1in,footskip=0.25in]{geometry}


\newcommand{\seqcl}[1]{{[#1]_{\text{seq}}}} 
%\pagestyle{fancy}
%\fancyhead{}
%\fancyfoot{}
%\lhead{} %Date
%\chead{} %CourseCode/Title
%\rhead{Nicolas Andrews}



\begin{document}

\section{Introduction}
\subsection{Outline of proposed research project}

The text Counterexamples in Topology by Steen and Seebach has been a fabulous resource for students and
researchers in Topology since its publication in 1970. The book was the product of an undergraduate research
project funded by NSF and supervised by Steen and Seebach (and including then student Gary Gruenhage) to
systematically survey important topological counterexamples. More recently James Dabbs has implemented a
database on Github based on the Steen and Seebach textbook called Pi-Base (see \href{https://topology.pi-base.org/}{https://topology.pi-base.org/})
and it is currently being maintained by Dabbs and Stephen Clontz. This resource has great potential to both
researchers and advanced undergraduate and graduate students at the start of their research careers. There are
still big gaps in the database's subject matter, especially in relation to research in and around Frechet-Urysohn
spaces. There is a significant body of work, and especially interesting counterexamples, concerning Michael's
class of bisequential spaces, Arhangel'skii's alpha-i spaces and several game theoretic formulations of
convergence which do not yet appear in the Pi-Base. The project has two goals. The first, and most accessible,
is to give a systematic survey of the recent research which will be implemented into the Pi-Base database. The
second half of the project will be devoted to open problems related to a recent class of examples defined from
ladder systems (more generally on so called square-sequence) described in two 

\section{Meeting Log}
\subsection*{Monday April 21} 
\begin{enumerate}
    \item Frechet Fan - \(S_{\omega}\): \(\omega \times (\omega + 1) / \omega \times \{\infty\}\) (i.e. \(\omega \times ( \omega + 1)\) with the points at infinity identified). Show that 
    \begin{itemize}
        \item \(S_{\omega}\) is not first countable \checkmark
        \item \(S_{\omega}\) is Fréchet. \checkmark
    \end{itemize}

    \item Product of Fréchet spaces not always Fréchet: take \((\omega + 1) \times S_{\omega}\). Let \(A = \{(m, (m, n)) \: : \: m,n \in \omega\). Show that 
    \begin{itemize}
        \item \((\omega + 1, \infty) \in \overline{A}\) \checkmark
        \item No sequence in \(A\) converges to \((\omega + 1, \infty)\). \checkmark
    \end{itemize}

    \item Right way to think about sequences: \(A \subseteq X\) converges to \(a \in X\) if \(|A| = \aleph_0\) and for all neighboourhoods \(U_x \subseteq X\), \(|A \setminus U_x| < \aleph_0\).
    
    \item Right way to think about Fréchet space: take sequential closure once same as closure.
    \item Another exercise: which \(\alpha_i\) properties does \(\times S_{\omega}\) have? \checkmark
\end{enumerate}
\subsection*{Thursday May 1}
\begin{enumerate}
    \item Go over example 3.9; why is \(S_{\omega}\) Fréchet?
    \item Give example of space that is \(\alpha_2\) but not \(\alpha_1\), \(\alpha_3\) but not \(\alpha_2\) etc. 
    \item Under which set theoretic assumptions do such examples exist? Look at the paper by Nyikos: Subsets of \(^\omega \omega\) and the Fréchet Urysohn and \(\alpha_i\) properties. 
    
    \item For example it is consistent that both \(\alpha_1\) and countable imply first countable, and \(\alpha_2\) and countable imply  \(\alpha_1\), yet there exists a countable \(\alpha_2\) space that is not first countable.

    \item We talked a little bit about topological groups (a topological space equipped with group operation that is continuous) and some of the nice structure they have: the group operation being continuous and invertible implies \(G\times G \to G\) is always a homeomorphism, in particular \(aG\) is the homeomorphic image of \(G\) by left multiplication of \(a \in G\), which in often cases lets us define a nbhd base at the identity and "send" it to all points of \(G\) in order to define the topology.
    \item Some more exercises 
    \begin{itemize}
        \item \(X\) is \(\alpha_1\) and Fréchet, \(Y\) is Fréchet, then \(X \times Y\) is Fréchet.
        \item \(\alpha_2\) is equivalent to \(A \cap B \neq \nothing\) for infinitely many \(A \in \xi\) whenever \(\xi\) is countable collection of sequences at \(x\).
    \end{itemize}
\end{enumerate}
%\section{Set Theory}
%\subsection{Ordinals}
%\subsection{Cardinals}
%\subsection{Ideals and Filters}
%
%\begin{defn}\cite{KU11}
%    Let \(X\) be a non empty set. We say that \(\I %\subseteq \P(X)\) is an \textit{ideal}  of \(X\) if 
%    \begin{enumerate}
%        \item[i.] \(\nothing \in \I \text{ and } X \not %\in \I\)
%        \item[ii.] \(U, V \in \I \implies U\cup V \in \I\)
%        \item[iii.] \(U \subset V\text{ and } V \in \I  %\implies U \in \I\)
%    \end{enumerate}
%\end{defn}
%
%\begin{defn}\cite{KU11}
%    Let \(X\) be a non empty set. We say that \(\F %\subseteq \P(X)\) is an \textit{filter}  of \(X\) if 
%    \begin{enumerate}
%        \item[i.] \( X \in \F \text{ and } \nothing \not %\in \F\)
%        \item[ii.] \(U, V \in \F \implies U\cap V \in \F\)
%        \item[iii.] \(U \subset V \text{ and } U \in \F  %\implies V \in \F\)
%    \end{enumerate}
%\end{defn}
%
%A filter that is maximal with respect to inclusion is %called am \textbf{ultrafilter}. As a result of Zorn's %lemma, every filter sits inside of an ultrafilter.


\section{Topology}

\subsection{Basics}
\begin{itemize}
    \item \(x \in \overline{A}\) iff \(U\cap A \neq \nothing\) for all open sets \(U\) containing \(x\).
\end{itemize}
    

\begin{defn}[Seperation Axioms]
    Let \(X\) be a topological space. \(X\) is said to be 
    \begin{itemize}
        \item \(T_0\): whenever \(x \neq y\) there exists an open \(U \subseteq X\) with \(x \in U\) but \(y \not \in U\) (or \(y \in U\) and \(x \not \in U\)).
        \item \(T_1\): whenever \(x \neq y\) there exists open sets \(U_x, U_y\) with \(x \not \in U_y\) and \(y \not \in U_x\)
        \item \(T_2\):  whenever \(x \neq y\) there exists open sets \(U_x, U_y\) with \(x \not \in U_y\) and \(y \not \in U_x\) and \(U_x \cap U_y = \nothing\)
        \item Regular: \(x \in X\) and closed \(A \subseteq X\) with \(x \not \in A\), there exists open disjoint \(U, V \subseteq X\) with \(x \in U\) and \(A \subseteq V\)
        \item \(T_3\): regular + \(T_1\)
        \item Completely regular: \(x \in X\) and closed \(A \subseteq X\) with \(x \not \in A\), there exists a continuous \(f: X \to \bR\) with \(f(x) = a\) and \(f(A) = b\) (\(a \neq b\))
        \item Tychonoff (\(T_{3\frac{1}{2}}\)): Completely regular + \(T_1\)
        \item Normal: Closed \(A, B \subseteq X\) with \(A \cap B = \nothing\), there exists open disjoint \(U, V \subseteq X\) with \(A \subseteq U\) and \(b \subseteq V\)
        \item Completely Normal: for all \(A \subseteq X\), \(A\) is normal.
    \end{itemize}
\end{defn}

\subsection{Sequential and Fréchet Spaces}

\begin{defn}
    For a topological space \(X\) and any set \(A \subset X\), the \textit{sequential closure} of \(A\) is  
    \[
        \seqcl{A} := \left\{x \in X \: : \: \exists (x_n) \in A \left(\lim_{n\to \infty} x_n = x\right) \right\}.
    \]
    In general we can repeat this operation recursively \(\seqcl{\seqcl{\seqcl{A}}\dots}\) by which is meant the \textit{total sequential closure} of \(A\).
    
\end{defn}

\begin{fact}
    In general it takes at most \(\omega_1\) many iterations of the sequential closure to get a closed set. 
\end{fact}

\begin{defn}
    A space \(X\) is said to be Fréchet if \(\seqcl{A} = \overline{A}\) for all \(A \subseteq X\).
\end{defn}
\begin{exam}
    Let \(X = \omega_1 + 1\) with the order topology. \(X\) is not Fréchet, since any sequence \((x_n) \in \omega_1\) cannot converge to \(\infty\), as otherwise \(\omega_1 = \sup\{x_n \: : \: n \in \bN\}\), a contradiction.
\end{exam}
\begin{defn}
    A space \(X\) is sequential if any closed set \(A \subseteq X\) is equal to its total sequential closure.
\end{defn}
Then a space that is Fréchet is also sequential. The following example shows that the converse is not true.

 
\begin{exam}
    Let \(X^{\ast} = \omega \times (\omega + 1)\) be given the order topology and let \(X =  X^{\ast} \cup \{\infty\}\) where the neighboourhoods  of \(\infty\) are such that there exists \(p \in \omega\) such that \(|\{(m, n) \: : \: m > p, n \in \omega + 1\} \setminus U_{\infty}| < \aleph_0\). Then \(X\) is sequential but but not Fréchet. To see this, note that for all \(m \in \omega\) the sequence \(A_m = \{(m, n) \: : \: n \in \omega\}\) converges to \((m, \omega + 1)\) and moreover \(B = \{(m, \omega + 1) \: : \: m \in \omega\}\) is a sequence that converges to \(\infty\). Then \(A = \bigcup_{m \in \omega} A_m\) is such that \(\seqcl{\seqcl{A}} = X\), hence \(X\) is sequential. On the other hand there is no sequence in \(A\) that converges to \(\infty\). Suppose there were, say some \(\gamma \to \infty\). Then for all \(m \in \omega\), \(U_m = X \setminus \{(m, n)\: : \: n \in \omega + 1\}\) is a neighboourhood of \(\infty\) such that \(|\gamma \setminus U_m| < \aleph_0\). Hence \(\gamma\) has only finitely many terms belonging to each column. If \(\alpha_m = \max\{\gamma \cap \{(m, n) \: : \: n \in \omega\}\}\), then \(U = X \setminus \bigcup_{m \in \omega} \{(m ,n )\: : \:  n \leq \alpha_m\}\) is a neighbourhood of \(\infty\) disjoint from \(\gamma\), a contradiction. Hence \(X\) is not Fréchet. 
\end{exam} 
\begin{prop}
    If \(X\) is first countable then \(X\) is Fréchet.
\end{prop}
\begin{proof}
    Let \(A \subseteq X\) and let \(x \in \overline{A}\). Then \(x\) has a countable neighbourhood base \(N_x\) such that \(U \cap A \neq \nothing\) for all \(U \in N_x\). Enumerating the neighboourhoods of \(x\) as \(U_1, U_2, \dots\) then the sequence \((x_n)_{n \ge 1}\) where \(x_n \in U_n \cap A\) for each \(n \in \omega\) is such that \((x_n)_{n \ge 1}\) converges to \(x\).
\end{proof}

The following example shows that the converse is not true. 
\begin{exam}[Fréchet Fan]
    Let \(S_{\omega}\) be the quotient of \(\omega \times (\omega + 1)\) obtained by identifying all the points \(\{(m, \omega + 1) \: : \: m \in \omega\}\) as \(\infty^{\ast}\). More precisely \(S_{\omega}\) has the quotient topology induced by the map \(h: \omega \times (\omega + 1) \to S_{\omega}\) where \(h(x) = x\) for all \(x \in \omega \times \omega\) and \(h(x) = \infty^{\ast}\) for all \(x \in \omega \times \{\omega + 1\}\). Then \(S_{\omega}\) is Fréchet but not 1st countable. To see that \(S_{\omega}\) is Fréchet, note that by definition of the quotient topology, the open neighbourhoods of \(\infty^{\ast}\) are those sets \(U \subset S_{\omega}\) such that \(\infty^{\ast} \in U\) and \(h^{-1}(U)\) is open in \(\omega \times (\omega + 1)\). As \(\{(m, \omega + 1) \: : \:  m \in \omega\} \subset h^{-1}(U)\) we see that \(U\) is an open neighbourhood of \(\infty^{\ast}\) iff \(h^{-1}(U)\) is an open neighbourhood of \((m, \omega + 1]\) for all \(m \in \omega\).  %Note that  \(h^{-1}(\{m\} \times (f(m), \infty^{\ast}]) = \{m\} \times (f(m), \omega + 1]\) is an open neighboourhood of \((m, \omega + 1]\) for each \(m \in \omega\) where \(f: \omega \to \omega\) is just some mapping that indicates the startpoint of each interval. 
    Hence the open neighboourhoods of \(\infty^{\ast}\) are of the form \(\bigcup_{m \in \omega}\{m\} \times (f(m), \infty^{\ast}]\) where \(f: \omega \to \omega\) is just some mapping that indicates the startpoint of each interval. It follows that for all \(m \in \omega\) the sequence \(A_m = \{(m, n) \: : \: n \in \omega\}\) converges to \(\infty^{\ast}\) so that \(A = \bigcup_{m \in \omega}A_m\) is such that \(\seqcl{A} = S_{\omega}\). On the other hand it's obvious that \(\overline{A} = S_{\omega}\), so that \(S_{\omega}\) is indeed Fréchet. Now asssuming that \(S_{\omega}\) was countable, we would have a countable neighbourhood base at \(\infty^{\ast}\). For each \(k \in \omega\) let \(B_k = \bigcup_{m \in \omega}\{m\} \times (f_k(m), \infty^{\ast}]\) for some \(f_k: \omega \to \omega\) determining the startpoints of each interval. Suppose \(\B = \{B_k \: : \: k \in \omega\}\) is a base at \(\infty^{\ast}\), then let \(f^{\ast}: \omega \to \omega\) be defined by \(f^{\ast}(m) = f_m(m) + 1\) for all \(m \in \omega\). Letting \(B^{\ast} = \bigcup_{m \in \omega}\{m\} \times (f^{\ast}(m), \infty^{\ast}]\) then \(B^{\ast}\) is an open neighboourhood of \(\infty^{\ast}\) but it is clear by construction that \(B_k \not\subset B^{\ast}\) for all \(k \in \omega\). Hence \(\B\) cannot be a neighboourhood base and   \(S_{\omega}\) is not first countable.
\end{exam}

As the following example shows, the product of Fréchet spaces need not be Fréchet.
\begin{exam}
    Let \(X = (\omega + 1) \times S_\omega\), and consider the set \(A = \{(m, (m, n))\: : \:  m, n \in \omega\}\). If \(\infty^{\ast}\) is the identified point of \(S_{\omega}\), let \(\infty = \{\omega + 1\} \times \infty^{\ast}\). Then \(\infty \in \overline{A}\) but \(\infty \not \in \seqcl{A}\). The open neighboourhoods of \(\infty\) are are of the form \((\alpha, \omega + 1] \times\left(\bigcup_{m \in \omega}\{m\} \times (f(m), \infty^{\ast}]\right)\), which clearly always has non emtpy intersection with \(A\). Hence \(\infty \in \overline{A}\). To see that \(\infty \not \in \seqcl{A}\), suppose \(\gamma\) is a sequence in \(A\) that converges to \(\infty\). Since the sets \(U_k = (k, \omega + 1] \times\left(\bigcup_{m \in \omega}\{m\} \times (f(m), \infty^{\ast}]\right)\) are open neighbourhoods of \(\infty\) it must be the case that \(|\gamma \setminus U_k | < \aleph_0\) for all \(k \in \omega\). Thus  \(\gamma\cap\left(\{k\}\times\{k\}\times(1,\omega + 1]\right)\) is finite for every \(k\). Let \(h:\omega \to \omega\) be defined by \(h(k) = \max\{\pi_3(\gamma\cap\left(\{k\}\times\{k\}\times(1,\infty^{\ast}]\right)\} + 1\) for \(k \in \omega\). Pictorially, \(h\) is picking the point on each spine beyond which no elements of \(\gamma\) exist. Thus 
    \[
    W = (1, \omega + 1] \times\left(\bigcup_{n \in \omega}\{n\}\times(h(n), \infty^{\ast}]\right)
    \] 
    %\[
    %W = \left(\bigcup_{m \in \omega} \bigcup_{n \in \omega}\{m\}\times\{n\}\times(h(n), \infty^{\ast}]\right)\cup\left(\{\omega + 1\}\times S_{\omega}\right)
    %\] 
    is an open neighbourhood of \(\infty\) which by construction is disjoint from \(\gamma\). Hence \(\gamma\) cannot converge to \(\infty\) showing that \(X\) is not Fréchet.
\end{exam}

\subsection{\(\alpha_i\) notions of convergence}

\begin{defn}
    Let \(X\) be a topological space and \(\xi\) be a countable family of sequences converging to a point \(x \in X\). We say that \(x\) is an \(\alpha_i\) point for \(i = 1, 2, 3, 4\) if there exists a sequence \(B\) such that 
    \begin{itemize}
        \item \(\alpha_1\):  \(|A \setminus B| < \aleph_0\) for every \(A \in \xi\);
        \item \(\alpha_2\): \(|A \cap B| = \aleph_0\) for every \(A \in \xi\);
        \item \(\alpha_3\): \(|A \cap B| = \aleph_0\) for infinitely many \(A \in \xi\);
        \item \(\alpha_4\): \(A \cap B \neq \nothing\) for infinitely many \(A \in \xi\).
    \end{itemize}
    Then \(X\) is an \(\alpha_i\) space if every \(x \in X\) is an \(\alpha_i\) point. Note that if a space is \(\alpha_i\) then it is \(\alpha_{i + 1}\) for \(i =1, 2, 3\).
\end{defn}

\begin{exam}
    \(S_{\omega}\) is not even \(\alpha_4\). For each \(m \in \omega\) let \(A_m = \{m\} \times (1, \infty)\). Then \(\xi = \{A_m \: : \: m \in \omega\}\) is a countable collection of sequences converging to \(\infty\). Suppose \(B\) is a sequence that converges to \(\infty\) such that \(A\cap B \neq \nothing\) for infinitely many \(A \in \xi\). In particular let \(\alpha \leq \omega\) be such that \(A_i \cap B \neq \nothing\) for all \(i \in \alpha\) and let \(f:\alpha \to \omega\) be defined by \(f(k) \in B \cap A_k\) for all \(k \in \alpha\). Then 
    \[
    U = \left(\bigcup_{k \in \alpha} \{k\} \times (f(k) + 1, \infty]\right)\times \left(\bigcup_{m \in \omega\setminus\alpha} \{m\} \times (1, \infty]\right)\] 
    is such that \(|B \cap U^C| = \aleph_0\), hence \(B\) does not converge to \(\infty\). 

    %\(S_{\omega}\) is \(\alpha_1\). Let \(x \in S_{\omega}\) and let \(\xi\) be a countable collection of sequences converging to \(x\). If \(x \neq \infty\), then since \(x\) is an isolated point any sequence converging to \(x\) will eventually be equal to \(x\), and therefore taking the constant sequence will satisfy the property. Otherwise, if \(x = \infty\), then 
    %\[
    %A = \bigcup_{m \in \omega}\bigcup_{1\leq i \leq m} \left(\xi_i \cap [1, \omega)\times \{m\}\right)
    %\] 
    %converges to \(\infty\) since we ensure that there are only finitely many elements per row (since each \(\xi_i\) can have only finitely many points in each row), and every sequence will appear in the union infinitely many times as we ascend. Since each \(\xi_i\) converges to \(\infty\) this guarantees that \(A\) will pick up infinitely many points of every  \(\xi_i\). 
\end{exam}



\subsection{Bisequential Spaces}

\begin{defn}
    
\end{defn}





%\subsection{Topological Games}
%\subsubsection{Two Player Convergence Game }
%    Let \(X\) be a topological space and designate a point %\(x_0 \in X\). The two player game is defined as %follows: 
%    \begin{itemize}
%            \item On turn one player I chooses an open set %\(U_1\) containing \(x_0\) and player II then %chooses a point \(x_1 \in U_1\);
%            \item On the \(n\)'th turn player I chooses an %open set \(U_n\) containing \(x_0\) and player %II then chooses a point \(x_n \in U_n\).
%    \end{itemize}  
%    Player I wins the game if the sequence \(x_n\) %converges to \(x_0\).
%
%
%
%
%
%\section{Examples}




%%%%%%%%%%%%%%%%%%%%%%%%%%%%%%%%%%%%%%%%%%%%%%%%%%%%%%%%%%%%%%%%%%%%%%%%%%%%%%%%%%%%%%%%%%%%%%%%%%%%%%%%%
%\newpage
%\bibliographystyle{plain}
%\bibliography{bibliography}{}

\end{document}