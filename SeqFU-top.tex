\documentclass{article}
\usepackage{texmacros}
\usepackage{titlesec}
\usepackage[margin=1in,footskip=0.25in]{geometry}


\newcommand{\seqcl}[1]{{[#1]_{\text{seq}}}} 
%\pagestyle{fancy}
%\fancyhead{}
%\fancyfoot{}
%\lhead{} %Date
%\chead{} %CourseCode/Title
%\rhead{Nicolas Andrews}



\begin{document}

\section{Introduction}
\subsection{Outline of proposed research project}

The text Counterexamples in Topology by Steen and Seebach has been a fabulous resource for students and
researchers in Topology since its publication in 1970. The book was the product of an undergraduate research
project funded by NSF and supervised by Steen and Seebach (and including then student Gary Gruenhage) to
systematically survey important topological counterexamples. More recently James Dabbs has implemented a
database on Github based on the Steen and Seebach textbook called Pi-Base (see \href{https://topology.pi-base.org/}{https://topology.pi-base.org/})
and it is currently being maintained by Dabbs and Stephen Clontz. This resource has great potential to both
researchers and advanced undergraduate and graduate students at the start of their research careers. There are
still big gaps in the database's subject matter, especially in relation to research in and around Frechet-Urysohn
spaces. There is a significant body of work, and especially interesting counterexamples, concerning Michael's
class of bisequential spaces, Arhangel'skii's alpha-i spaces and several game theoretic formulations of
convergence which do not yet appear in the Pi-Base. The project has two goals. The first, and most accessible,
is to give a systematic survey of the recent research which will be implemented into the Pi-Base database. The
second half of the project will be devoted to open problems related to a recent class of examples defined from
ladder systems (more generally on so called square-sequence) described in two 

\section{Meeting Log}
\subsection*{Monday April 21} 
\begin{enumerate}
    \item Frechet Fan - \(S_{\omega}\): \(\omega \times (\omega + 1) / \omega \times \{\infty\}\) (i.e. \(\omega \times ( \omega + 1)\) with the points at infinity identified). Show that 
    \begin{itemize}
        \item \(S_{\omega}\) is not first countable \checkmark
        \item \(S_{\omega}\) is Fréchet. \checkmark
    \end{itemize}

    \item Product of Fréchet spaces not always Fréchet: take \((\omega + 1) \times S_{\omega}\). Let \(A = \{(m, (m, n)) \: : \: m,n \in \omega\). Show that 
    \begin{itemize}
        \item \((\omega + 1, \infty) \in \overline{A}\) \checkmark
        \item No sequence in \(A\) converges to \((\omega + 1, \infty)\). \checkmark
    \end{itemize}

    \item Right way to think about sequences: \(A \subseteq X\) converges to \(a \in X\) if \(|A| = \aleph_0\) and for all neighboourhoods \(U_x \subseteq X\), \(|A \setminus U_x| < \aleph_0\).
    
    \item Right way to think about Fréchet space: take sequential closure once same as closure.
    \item Another exercise: which \(\alpha_i\) properties does \(S_{\omega}\) have? \checkmark
\end{enumerate}
\subsection*{Thursday May 1}
\begin{enumerate}
    \item Go over example 3.9; why is \(S_{\omega}\) Fréchet? \checkmark
    \item Give example of space that is \(\alpha_2\) but not \(\alpha_1\) \checkmark, \(\alpha_3\) but not \(\alpha_2\) etc. 
    \item Under which set theoretic assumptions do such examples exist? Look at the paper by Nyikos: Subsets of \(^\omega \omega\) and the Fréchet Urysohn and \(\alpha_i\) properties. \textcolor{red}{Good grasp at sections 1, 2, 5; partial understanding section 3; skipped section 4, 6}.
    
    \item For example it is consistent that both \(\alpha_1\) and countable imply first countable, and \(\alpha_2\) and countable imply  \(\alpha_1\), yet there exists a countable \(\alpha_2\) space that is not first countable.

    \item We talked a little bit about topological groups (a topological space equipped with group operation that is continuous) and some of the nice structure they have: the group operation being continuous and invertible implies \(G\times G \to G\) is always a homeomorphism, in particular \(aG\) is the homeomorphic image of \(G\) by left multiplication of \(a \in G\), which in often cases lets us define a nbhd base at the identity and "send" it to all points of \(G\) in order to define the topology.
    \item Look into what it takes to contribute to the pi-base, in particular add the \(\alpha_i\) spaces.
    \textcolor{red}{Forked database to github account; properties are stored as markdown text files, should not be difficult to add properties as well as if/then.}

    Things to add:
    \begin{itemize}
        \item \(\alpha_i\) and if \(\alpha_i\) then \(\alpha_{i + 1}\)
        \item bisequential
        \item v-space (w-space exists already)
        \item if 1st countable then \(\alpha_1\) Frechet
        \item if bisequential then Frechet
        \item w-space iff Frechet and \(\alpha_2\)   
    \end{itemize}
    \item Some more exercises 
    \begin{itemize}
        \item \(X\) is \(\alpha_1\) and Fréchet, \(Y\) is Fréchet, then \(X \times Y\) is Fréchet. \textcolor{red}{this is not true, take $(\omega + 1) \times S_{\omega}$.}
        \item \(\alpha_2\) is equivalent to \(A \cap B \neq \nothing\) for all \(A \in \xi\) whenever \(\xi\) is countable collection of sequences at \(x\). \checkmark
    \end{itemize}
\end{enumerate}
\subsection*{Thursday May 15}
\begin{enumerate}
    \item When is the square of an \(\alpha_i\) space still \(\alpha_i\)? What \(\alpha\) properties does \(\alpha_i \times \alpha_j\) have?
    \item I asked for an example of a tower, and Paul proved that towers exist, but said that theres not really specific constructions of towers.
    \item historical note: the cardinals that lay between \(\omega_1\) and \(\P(\omega)\) came from attempts to solve CH, for example, if \(\mathfrak{t} \leq \mathfrak{c}\) then, finding the cardinality for \(\mathfrak{t}\) would help in finding cardinality of \(\mathfrak{c}\).
    \item Proximal game
    \item Uniformities
    \item Homogeneity in Top group used to determine uniformity for player 1 in prox game
    \item FUF and 2FUF
    \item exercises:
    \begin{itemize}
        \item prove lemmas used in existence proof of towers.
        \item If \(\{A_{\alpha} \: : \: \alpha \subseteq \gamma \}\) is a tower and \(\{\alpha_{\xi} \: : \: \xi < \lambda\}\) is increasing and cofinal in \(\gamma\) then \(A_{\alpha_\xi} \: : \: \xi < \lambda\) is a tower.
        \item \(S_{\omega}\) is neither FUF or \(2FUF\)
        \item Proximal game is equivalent to Gruenhage game.
    \end{itemize}
    \item note: send Paul changes to databse before pushing to github
\end{enumerate}

\section{Set Theory}
%\subsection{Ordinals and Cardinal}
%\begin{defn}
%    \(\aleph_1 = \omega_1\) (not definition but kind of?)
%\end{defn}
%\begin{thm}
%    \(|\P(\omega)| = |\bR|\).
%\end{thm}
%\begin{defn}[Continuum Hypothesis]
%    CH is the statement \(2^{\aleph_0} = \aleph_1\).
%\end{defn}
%CH is indepenet of ZFC: first shown by Gödel to not be inconsistent with ZFC and later shown by Cohen that its %negation is also consistent. Alternatively we can formulatie of CH as: \(\omega_1 = \P(\omega) = \mathfrak{c}\) %where \(\mathfrak{c} = |\bR|\).

\subsection{Some Interesting Cardinals}

\begin{defn}
    Let \(\A\) be an infinite family of infinite subsets of \(\omega\). The family \(\A\) is said to be an almost disjoint family (ADF) of subsets of \(\omega\) if \(A \cap B \) is finite for any \(A,B \in \mathcal{A}\) with \(A \ne B\). By Zorn's lemma we can assume \(\A\) lives inside of a maximal (w.r.t. inclusion) family that is also pairwise disjoint. Such a family is a maximal almost disjoint family (MADF).
\end{defn}
\begin{defn}
    Define an order \(\leq^{\ast}\) on \(^\omega\omega\) where \(f <^{\ast} g\) iff there exists \(n\) such that \(f(k) < g(k)\) for all \(k \geq n\)
\end{defn}
\begin{defn}
    Let \((A, \prec)\) be a linearly ordered set. Then we say that \(B \subseteq A\) is undominating in \(A\) if there does not exist \(y \in A\) such that \(x \prec y\) for all \(x \in B\). We say that \(C \subseteq A\) is cofinal (or dominating) in \(A\) if for all \(x \in A\) there exists \(z \in C\) such that \(x \prec x\).
\end{defn}

\begin{defn}
    \leavevmode
    \begin{itemize}
        \item \(\mathfrak{a} = \min\{|\A| \: : \: \A \text{ is infinte } \text{MADF}\}\)
        \item \(\mathfrak{b} = \min\{|F| \: : \: F \text{ is } \leq^{\ast}\text{-undominated subset of }^\omega\omega\}\)
        \item \(\mathfrak{d} = \min\{|F| \: : \: F \text{ is } \leq^{\ast}\text{-cofinal in } ^\omega\omega
        \}\)
    \end{itemize}
    Each of \(\mathfrak{a}, \mathfrak{b}, \mathfrak{d}\) are at least \(\omega_1\) but at most \(\P(\omega)\).
\end{defn}


\section{Topology}

\subsection{Basics}

\subsection{Sequential and Fréchet Spaces}

\begin{defn}
    For a topological space \(X\) and any set \(A \subset X\), the \textit{sequential closure} of \(A\) is  
    \[
        \seqcl{A} := \left\{x \in X \: : \: \exists (x_n) \in A \left(\lim_{n\to \infty} x_n = x\right) \right\}.
    \]
    In general we can repeat this operation recursively \(\seqcl{\seqcl{\seqcl{A}}\dots}\) by which is meant the \textit{total sequential closure} of \(A\).
    
\end{defn}

\begin{fact}
    In general it takes at most \(\omega_1\) many iterations of the sequential closure to get a closed set. 
\end{fact}

\begin{defn}
    A space \(X\) is said to be Fréchet if \(\seqcl{A} = \overline{A}\) for all \(A \subseteq X\).
\end{defn}
\begin{exam}
    Let \(X = \omega_1 + 1\) with the order topology. \(X\) is not Fréchet, since any sequence \((x_n) \in \omega_1\) cannot converge to \(\infty\), as otherwise \(\omega_1 = \sup\{x_n \: : \: n \in \bN\}\), a contradiction.
\end{exam}
\begin{defn}
    A space \(X\) is sequential if any closed set \(A \subseteq X\) is equal to its total sequential closure.
\end{defn}
Then a space that is Fréchet is also sequential. The following example shows that the converse is not true.

 
\begin{exam}
    Let \(X^{\ast} = \omega \times (\omega + 1)\) be given the order topology and let \(X =  X^{\ast} \cup \{\infty\}\) where the neighboourhoods  of \(\infty\) are such that there exists \(p \in \omega\) such that \(|\{(m, n) \: : \: m > p, n \in \omega + 1\} \setminus U_{\infty}| < \aleph_0\). Then \(X\) is sequential but but not Fréchet. To see this, note that for all \(m \in \omega\) the sequence \(A_m = \{(m, n) \: : \: n \in \omega\}\) converges to \((m, \omega + 1)\) and moreover \(B = \{(m, \omega + 1) \: : \: m \in \omega\}\) is a sequence that converges to \(\infty\). Then \(A = \bigcup_{m \in \omega} A_m\) is such that \(\seqcl{\seqcl{A}} = X\), hence \(X\) is sequential. On the other hand there is no sequence in \(A\) that converges to \(\infty\). Suppose there were, say some \(\gamma \to \infty\). Then for all \(m \in \omega\), \(U_m = X \setminus \{(m, n)\: : \: n \in \omega + 1\}\) is a neighboourhood of \(\infty\) such that \(|\gamma \setminus U_m| < \aleph_0\). Hence \(\gamma\) has only finitely many terms belonging to each column. If \(\alpha_m = \max\{\gamma \cap \{(m, n) \: : \: n \in \omega\}\}\), then \(U = X \setminus \bigcup_{m \in \omega} \{(m ,n )\: : \:  n \leq \alpha_m\}\) is a neighbourhood of \(\infty\) disjoint from \(\gamma\), a contradiction. Hence \(X\) is not Fréchet. 
\end{exam} 
\begin{prop}
    If \(X\) is first countable then \(X\) is Fréchet.
\end{prop}
\begin{proof}
    Let \(A \subseteq X\) and let \(x \in \overline{A}\). Then \(x\) has a countable neighbourhood base \(N_x\) such that \(U \cap A \neq \nothing\) for all \(U \in N_x\). Enumerating the neighboourhoods of \(x\) as \(U_1, U_2, \dots\) then the sequence \((x_n)_{n \ge 1}\) where \(x_n \in U_n \cap A\) for each \(n \in \omega\) is such that \((x_n)_{n \ge 1}\) converges to \(x\).
\end{proof}

The following example shows that the converse is not true. 
\begin{exam}[Fréchet Fan]
    Let \(S_{\omega}\) be the quotient of \(\omega \times (\omega + 1)\) obtained by identifying all the points \(\{(m, \omega + 1) \: : \: m \in \omega\}\) as \(\infty^{\ast}\). More precisely \(S_{\omega}\) has the quotient topology induced by the map \(h: \omega \times (\omega + 1) \to S_{\omega}\) where \(h(x) = x\) for all \(x \in \omega \times \omega\) and \(h(x) = \infty^{\ast}\) for all \(x \in \omega \times \{\omega + 1\}\). Then \(S_{\omega}\) is Fréchet but not 1st countable. 
    
    To see that \(S_{\omega}\) is Fréchet, let \(A \subseteq S_{\omega}\) such that \(\infty \in \overline{A}\). \(A\) must meet at least one column of \(S_{\omega}\) in an infinite set, otherwise we could find a nbhd of \(\infty^{\ast}\) disjoint from \(A\). Then \(A\) restricted to that column will be a convergent sequence to \(\infty\).
    
    Now asssuming that \(S_{\omega}\) was countable, we would have a countable neighbourhood base at \(\infty^{\ast}\). For each \(k \in \omega\) let \(B_k = \bigcup_{m \in \omega}\{m\} \times (f_k(m), \infty^{\ast}]\) for some \(f_k: \omega \to \omega\) determining the startpoints of each interval. Suppose \(\B = \{B_k \: : \: k \in \omega\}\) is a base at \(\infty^{\ast}\), then let \(f^{\ast}: \omega \to \omega\) be defined by \(f^{\ast}(m) = f_m(m) + 1\) for all \(m \in \omega\). Letting \(B^{\ast} = \bigcup_{m \in \omega}\{m\} \times (f^{\ast}(m), \infty^{\ast}]\) then \(B^{\ast}\) is an open neighboourhood of \(\infty^{\ast}\) but it is clear by construction that \(B_k \not\subset B^{\ast}\) for all \(k \in \omega\). Hence \(\B\) cannot be a neighboourhood base and   \(S_{\omega}\) is not first countable.
\end{exam}

As the following example shows, the product of Fréchet spaces need not be Fréchet.
\begin{exam}
    Let \(X = (\omega + 1) \times S_\omega\), and consider the set \(A = \{(m, (m, n))\: : \:  m, n \in \omega\}\). If \(\infty^{\ast}\) is the identified point of \(S_{\omega}\), let \(\infty = \{\omega + 1\} \times \infty^{\ast}\). Then \(\infty \in \overline{A}\) but \(\infty \not \in \seqcl{A}\). The open neighboourhoods of \(\infty\) are are of the form \((\alpha, \omega + 1] \times\left(\bigcup_{m \in \omega}\{m\} \times (f(m), \infty^{\ast}]\right)\), which clearly always has non emtpy intersection with \(A\). Hence \(\infty \in \overline{A}\). To see that \(\infty \not \in \seqcl{A}\), suppose \(\gamma\) is a sequence in \(A\) that converges to \(\infty\). Since the sets \(U_k = (k, \omega + 1] \times\left(\bigcup_{m \in \omega}\{m\} \times (f(m), \infty^{\ast}]\right)\) are open neighbourhoods of \(\infty\) it must be the case that \(|\gamma \setminus U_k | < \aleph_0\) for all \(k \in \omega\). Thus  \(\gamma\cap\left(\{k\}\times\{k\}\times(1,\omega + 1]\right)\) is finite for every \(k\). Let \(h:\omega \to \omega\) be defined by \(h(k) = \max\{\pi_3(\gamma\cap\left(\{k\}\times\{k\}\times(1,\infty^{\ast}]\right)\} + 1\) for \(k \in \omega\). Pictorially, \(h\) is picking the point on each spine beyond which no elements of \(\gamma\) exist. Thus 
    \[
    W = (1, \omega + 1] \times\left(\bigcup_{n \in \omega}\{n\}\times(h(n), \infty^{\ast}]\right)
    \] 
    %\[
    %W = \left(\bigcup_{m \in \omega} \bigcup_{n \in \omega}\{m\}\times\{n\}\times(h(n), \infty^{\ast}]\right)\cup\left(\{\omega + 1\}\times S_{\omega}\right)
    %\] 
    is an open neighbourhood of \(\infty\) which by construction is disjoint from \(\gamma\). Hence \(\gamma\) cannot converge to \(\infty\) showing that \(X\) is not Fréchet.
\end{exam}

\subsection{\(\alpha_i\) notions of convergence}

\begin{defn}
    Let \(X\) be a topological space and \(\xi\) be a countable family of sequences converging to a point \(x \in X\). We say that \(x\) is an \(\alpha_i\) point for \(i = 1, 2, 3, 4\) if there exists a sequence \(B\) such that 
    \begin{itemize}
        \item \(\alpha_1\):  \(|A \setminus B| < \aleph_0\) for every \(A \in \xi\);
        \item \(\alpha_2\): \(|A \cap B| = \aleph_0\) for every \(A \in \xi\);
        \item \(\alpha_3\): \(|A \cap B| = \aleph_0\) for infinitely many \(A \in \xi\);
        \item \(\alpha_4\): \(A \cap B \neq \nothing\) for infinitely many \(A \in \xi\).
    \end{itemize}
    Then \(X\) is an \(\alpha_i\) space if every \(x \in X\) is an \(\alpha_i\) point. Note that if a space is \(\alpha_i\) then it is \(\alpha_{i + 1}\) for \(i =1, 2, 3\).
\end{defn}

\begin{exam}
    \(S_{\omega}\) is not even \(\alpha_4\). For each \(m \in \omega\) let \(A_m = \{m\} \times (1, \infty)\). Then \(\xi = \{A_m \: : \: m \in \omega\}\) is a countable collection of sequences converging to \(\infty\). Suppose \(B\) is a sequence that converges to \(\infty\) such that \(A\cap B \neq \nothing\) for infinitely many \(A \in \xi\). In particular let \(\alpha \leq \omega\) be such that \(A_i \cap B \neq \nothing\) for all \(i \in \alpha\) and let \(f:\alpha \to \omega\) be defined by \(f(k) \in B \cap A_k\) for all \(k \in \alpha\). Then 
    \[
    U = \left(\bigcup_{k \in \alpha} \{k\} \times (f(k) + 1, \infty]\right)\times \left(\bigcup_{m \in \omega\setminus\alpha} \{m\} \times (1, \infty]\right)\] 
    is such that \(|B \cap U^C| = \aleph_0\), hence \(B\) does not converge to \(\infty\). 

\end{exam}

\begin{exercise}
    X is \(\alpha_2\) iff whenever \(\xi\) is a countable collection of sequences converging to \(x\) there exists \(B \to x\) such \(A \cap B \neq \nothing\) for all many \(A \in \xi\).
\end{exercise}
\begin{soln}
    The forward direction is obvious. Conversely, let \(\xi =\{A_1, A_2, \dots\}\) be a countable collection of sequences converging to \(x\). For every \(n\in \omega\) let \(\{a_{nm} \: : \: m \in \omega\}\) enumerate the elements of \(A_n\) and define \(A_{nm} = A_n \setminus \{a_{n1}, a_{n2}, \dots, a_{n,m-1}\}\) for each \(n,m \in \omega\). Then the \(A_{nm}\) still convege to \(x\) and  \(\A = \{A_{nm} \: : \: n,m \in \omega \}\) is a sheaf at \(x\). By hypothesis there exists a \(B\) converging to \(x\) that meets each \(A_{nm}\). Thus \(B\) meets each \(A_n\) in an infinite set.
\end{soln}

\subsection{\(\Psi\)-like spaces}

\begin{defn}
    Let \(D\) be a countable set and let \(\A \subseteq \P(D)\) be an ADF. Define a topology on \(X = D \cup \A\) such that the points of \(D\) are isolated and \(D\) is dense in \(X\). Then for every \(A \in \A\) attach a point \(z_A\) where the nbhds of \(z_A\) consist of all sets of the form \(B(A, F) = \{z_A\} \cup (A \setminus F)\) where \(F \subseteq A\) is finite.
\end{defn}

\begin{prop}
    Let \(X\) be as above. Then \(X\) is Hausdorff, first countable and locally compact. If \(X\) is constructed as above but \(\A\) is MADF then \(X\) is also pseudo-compact (i.e. all continuous real functions on \(X\) are bounded).
\end{prop}

\begin{proof}
    To check Hausdorffness it suffices to show \(Z = X\setminus D\) is Hausdorff. Let \(z_1, z_2 \in Z\). Then \(z_1, z_2\) correspond to sets \(A_1, A_2\) such that \(A_1 \cap A_2 = I\) is finite. Then \(B(A_1, I)\) and \(B(A_2, I)\) are disjoint open sets containing \(z_1, z_2\) respectively. 

    To see that \(X\) is locally compact, note that \(\{x\}\) is a compact neighbourhood base for all \(x \in D\). Local compactness of \(Z\) follows from the fact that by design each \(z_A \in Z\) is the limit of \(A\) viewed as a sequence. In other words, whenever \(B(A_z, F)\) is a basic nbhd of \(z_A\), any open cover will have a smaller nbhd \(B(A_z, F')\) that omits only finitely many points.
\end{proof}

\begin{prop}
    The one point compactification of a \(\Psi\)-like space of cardinality \(<\mathfrak{a}\) is Fréchet. 
\end{prop}

\subsection{Constructing Examples of \(\alpha_i\) Spaces}

\begin{exam}
     Let \(D = \omega\times\omega\), let \(\F = \{f_{\alpha} \: : \: \alpha < \mathfrak{b}\} \subseteq \, ^\omega\omega\) be \(<^{\ast}\)-well-ordered and \(<^{\ast}\)-unbounded, and let \(\C = \{C_n = \{n\} \times \omega \: : \: n \in \omega\}\). Construct a \(\Psi\)-like from \(X = \F \cup \C\), where we attach compactification points \(\{n\} \times (\omega+1)\) for each \(C_n\) and compactification points \(p_{\alpha}\) for each \(f_{\alpha}\). Then the resulting one point compactification, \(X + \infty\), is \(\alpha_2\) and Fréchet but not \(\alpha_1\). \(X + \infty\) becomes \(\alpha_1\) if \(\F\) is cofinal and well ordered. 
\end{exam}

\subsection{Bisequential Spaces}

\begin{defn}
    
\end{defn}



%\subsection{Topological Games}
%\subsubsection{Two Player Convergence Game }
%    Let \(X\) be a topological space and designate a point %\(x_0 \in X\). The two player game is defined as %follows: 
%    \begin{itemize}
%            \item On turn one player I chooses an open set %\(U_1\) containing \(x_0\) and player II then %chooses a point \(x_1 \in U_1\);
%            \item On the \(n\)'th turn player I chooses an %open set \(U_n\) containing \(x_0\) and player %II then chooses a point \(x_n \in U_n\).
%    \end{itemize}  
%    Player I wins the game if the sequence \(x_n\) %converges to \(x_0\).
%
%
%
%
%
%\section{Examples}




%%%%%%%%%%%%%%%%%%%%%%%%%%%%%%%%%%%%%%%%%%%%%%%%%%%%%%%%%%%%%%%%%%%%%%%%%%%%%%%%%%%%%%%%%%%%%%%%%%%%%%%%%
%\newpage
%\bibliographystyle{plain}
%\bibliography{bibliography}{}

\end{document}