\documentclass{article}
\usepackage{texmacros}
\usepackage{titlesec}
\usepackage[margin=1in,footskip=0.25in]{geometry}


\newcommand{\seqcl}[1]{{[#1]_{\text{seq}}}} 
\newcommand{\nothing}{\rotatebox[origin=c]{23.8}{$\varnothing$}}
%\pagestyle{fancy}
%\fancyhead{}
%\fancyfoot{}
%\lhead{} %Date
%\chead{} %CourseCode/Title
%\rhead{Nicolas Andrews}



\begin{document}

\section{Introduction}
\subsection{Outline of proposed research project}

The text Counterexamples in Topology by Steen and Seebach has been a fabulous resource for students and
researchers in Topology since its publication in 1970. The book was the product of an undergraduate research
project funded by NSF and supervised by Steen and Seebach (and including then student Gary Gruenhage) to
systematically survey important topological counterexamples. More recently James Dabbs has implemented a
database on Github based on the Steen and Seebach textbook called Pi-Base (see \href{https://topology.pi-base.org/}{https://topology.pi-base.org/})
and it is currently being maintained by Dabbs and Stephen Clontz. This resource has great potential to both
researchers and advanced undergraduate and graduate students at the start of their research careers. There are
still big gaps in the database's subject matter, especially in relation to research in and around Frechet-Urysohn
spaces. There is a significant body of work, and especially interesting counterexamples, concerning Michael's
class of bisequential spaces, Arhangel'skii's alpha-i spaces and several game theoretic formulations of
convergence which do not yet appear in the Pi-Base. The project has two goals. The first, and most accessible,
is to give a systematic survey of the recent research which will be implemented into the Pi-Base database. The
second half of the project will be devoted to open problems related to a recent class of examples defined from
ladder systems (more generally on so called square-sequence) described in two 


\section{Set Theory}
\subsection{Ordinals}
\subsection{Cardinals}
\subsection{Ideals and Filters}

\section{Topology}

\subsection{Sequential and Fréchet-Urysohn Spaces}

\begin{defn}\cite{EK89}
    A space \(X\) is \textit{sequential} if 
    \[
    \displaystyle A \subset X \text{ is closed } \iff \forall (x_n) \in A\left(\lim_{n\to \infty} x_n \in A\right).
    \] 
\end{defn}

\begin{defn}\cite{AH90}
    For a topological space \(X\) and any set \(A \subset X\), the \textit{sequential closure of} \(A\) is  
    \[
        \seqcl{A} := \left\{x \in X \: : \: \exists (x_n) \in A \left(\lim_{n\to \infty} x_n = x\right) \right\}
    \]

\end{defn}

\begin{defn}\cite{NY92}
    Let \(X\) be a topological space and \(\varsigma\) be a countable family of sequences converging to a point \(x_0 \in X\). We say that \(x_0\) is an \(\alpha_i\) point for \(i = 1, 2, 3, 4\) if there exists a sequence \(\beta\) such that 
    \begin{itemize}
        \item \(\alpha_1\):  \(|\gamma \setminus \beta| < \aleph_0\) for every \(\gamma \in \varsigma\);
        \item \(\alpha_2\): \(|\gamma \cap \beta| = \aleph_0\) for every \(\gamma \in \varsigma\);
        \item \(\alpha_3\): \(|\gamma \cap \beta| = \aleph_0\) for infinitely many \(\gamma \in \varsigma\);
        \item \(\alpha_4\): \(\gamma \cap \beta \neq \nothing\) for infinitely many \(\gamma \in \varsigma\).
    \end{itemize}
    Then \(X\) is an \(\alpha_i\) space if every \(x \in X\) is an \(\alpha_i\) point.
\end{defn}





\subsection{Bisequential Spaces}

\begin{defn}
    Let \(X\) be a topological space. If for all \(x \in X\) and \(\mathcal{F} \subset \P(X)\) an ultrafilter which clusters at \(x\) there exists a sequence of sets \(A_1 \supseteq A_2 \supseteq A_3 \dots\) where each \(A_i \in \F\)
\end{defn}





\subsection{Topological Games}
\subsubsection{Two Player Convergence Game (\texorpdfstring{\cite{GH76}})}
    Let \(X\) be a topological space and designate a point \(x_0 \in X\). The two player game is defined as follows: 
    \begin{itemize}
            \item On turn one player I chooses an open set \(U_1\) containing \(x_0\) and player II then chooses a point \(x_1 \in U_1\);
            \item On the \(n\)'th turn player I chooses an open set \(U_n\) containing \(x_0\) and player II then chooses a point \(x_n \in U_n\).
    \end{itemize}  
    Player I wins the game if the sequence \(x_n\) converges to \(x_0\).


    \cite{IF13}



\section{Examples}

\begin{exam}\cite{WL04}
    Let \(X = \omega_1 + \{\infty\}\) with the order topology. \(X\) is not sequential, for consider \(\omega_1 \in X\), which has \(\overline{\omega_1} \setminus \omega_1 = \{\infty\}\). Then by definition, any sequence \((x_n) \in \omega_1\) cannot converge to \(\infty\), as otherwise \(\omega_1 = \sup\{x_n \: : \: n \in \bN\}\), a contradiction.
\end{exam}

\begin{exam}{Aren's Space}
    
\end{exam}



%%%%%%%%%%%%%%%%%%%%%%%%%%%%%%%%%%%%%%%%%%%%%%%%%%%%%%%%%%%%%%%%%%%%%%%%%%%%%%%%%%%%%%%%%%%%%%%%%%%%%%%%%
\newpage
\bibliographystyle{plain}
\bibliography{bibliography}{}

\end{document}