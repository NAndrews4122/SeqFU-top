\documentclass{article}
\usepackage{texmacros}
\usepackage{titlesec}
\usepackage[margin=1in,footskip=0.25in]{geometry}



\newcommand{\seqcl}[1]{{[#1]_{\text{seq}}}} 
%\pagestyle{fancy}
%\fancyhead{}
%\fancyfoot{}
%\lhead{} %Date
%\chead{} %CourseCode/Title
%\rhead{Nicolas Andrews}



\begin{document}
\section{Introduction}
\subsection{Outline of proposed research project}

The text Counterexamples in Topology by Steen and Seebach has been a fabulous resource for students and
researchers in Topology since its publication in 1970. The book was the product of an undergraduate research
project funded by NSF and supervised by Steen and Seebach (and including then student Gary Gruenhage) to
systematically survey important topological counterexamples. More recently James Dabbs has implemented a
database on Github based on the Steen and Seebach textbook called Pi-Base (see \href{https://topology.pi-base.org/}{https://topology.pi-base.org/})
and it is currently being maintained by Dabbs and Stephen Clontz. This resource has great potential to both
researchers and advanced undergraduate and graduate students at the start of their research careers. There are
still big gaps in the database's subject matter, especially in relation to research in and around Frechet-Urysohn
spaces. There is a significant body of work, and especially interesting counterexamples, concerning Michael's
class of bisequential spaces, Arhangel'skii's alpha-i spaces and several game theoretic formulations of
convergence which do not yet appear in the Pi-Base. The project has two goals. The first, and most accessible,
is to give a systematic survey of the recent research which will be implemented into the Pi-Base database. The
second half of the project will be devoted to open problems related to a recent class of examples defined from
ladder systems (more generally on so called square-sequence) described in two 

\section{Meeting Log}
\subsection*{Monday April 21} 
\begin{enumerate}
    \item Frechet Fan - \(S_{\omega}\): \(\omega \times (\omega + 1) / \omega \times \{\infty\}\) (i.e. \(\omega \times ( \omega + 1)\) with the points at infinity identified). Show that 
    \begin{itemize}
        \item \(S_{\omega}\) is not first countable \checkmark
        \item \(S_{\omega}\) is Fréchet. \checkmark
    \end{itemize}

    \item Product of Fréchet spaces not always Fréchet: take \((\omega + 1) \times S_{\omega}\). Let \(A = \{(m, (m, n)) \: : \: m,n \in \omega\). Show that 
    \begin{itemize}
        \item \((\omega + 1, \infty) \in \overline{A}\) \checkmark
        \item No sequence in \(A\) converges to \((\omega + 1, \infty)\). \checkmark
    \end{itemize}

    \item Right way to think about sequences: \(A \subset X\) converges to \(a \in X\) if \(|A| = \aleph_0\) and for all neighboourhoods \(U_x \subset X\), \(|A \setminus U_x| < \aleph_0\).
    
    \item Right way to think about Fréchet space: take sequential closure once same as closure.
    \item Another exercise: which \(\alpha_i\) properties does \(S_{\omega}\) have? \checkmark
\end{enumerate}
\subsection*{Thursday May 1}
\begin{enumerate}
    \item Go over example 3.9; why is \(S_{\omega}\) Fréchet? \checkmark
    \item Give example of space that is \(\alpha_2\) but not \(\alpha_1\) \checkmark, \(\alpha_3\) but not \(\alpha_2\) etc. 
    \item Under which set theoretic assumptions do such examples exist? Look at the paper by Nyikos: Subsets of \(^\omega \omega\) and the Fréchet Urysohn and \(\alpha_i\) properties. \textcolor{red}{Good grasp at sections 1, 2, 5; partial understanding section 3; skipped section 4, 6}.
    
    \item For example it is consistent that both \(\alpha_1\) and countable imply first countable, and \(\alpha_2\) and countable imply  \(\alpha_1\), yet there exists a countable \(\alpha_2\) space that is not first countable.

    \item We talked a little bit about topological groups (a topological space equipped with group operation that is continuous) and some of the nice structure they have: the group operation being continuous and invertible implies \(G\times G \to G\) is always a homeomorphism, in particular \(aG\) is the homeomorphic image of \(G\) by left multiplication of \(a \in G\), which in often cases lets us define a nbhd base at the identity and "send" it to all points of \(G\) in order to define the topology.
    \item Look into what it takes to contribute to the pi-base, in particular add the \(\alpha_i\) spaces.
    \textcolor{red}{Forked database to github account; properties are stored as markdown text files, should not be difficult to add properties as well as if/then.}

    Things to add:
    \begin{itemize}
        \item \(\alpha_i\) and if \(\alpha_i\) then \(\alpha_{i + 1}\)
        \item bisequential
        \item v-space (w-space exists already)
        \item if 1st countable then \(\alpha_1\) Frechet
        \item if bisequential then Frechet
        \item w-space iff Frechet and \(\alpha_2\)   
    \end{itemize}
    \item Some more exercises 
    \begin{itemize}
        \item \(X\) is \(\alpha_1\) and Fréchet, \(Y\) is Fréchet, then \(X \times Y\) is Fréchet. \textcolor{red}{this is not true, take $(\omega + 1) \times S_{\omega}$.}
        \item \(\alpha_2\) is equivalent to \(A \cap B \neq \nothing\) for all \(A \in \xi\) whenever \(\xi\) is countable collection of sequences at \(x\). \checkmark
    \end{itemize}
\end{enumerate}
\subsection*{Thursday May 15}
\begin{enumerate}
    \item When is the square of an \(\alpha_i\) space still \(\alpha_i\)? What \(\alpha\) properties does \(\alpha_i \times \alpha_j\) have?
    \item I asked for an example of a tower, and Paul proved that towers exist, but said that theres not really specific constructions of towers.
    \item historical note: the cardinals that lay between \(\omega_1\) and \(\P(\omega)\) came from attempts to solve CH, for example, if \(\mathfrak{t} \leq \mathfrak{c}\) then, finding the cardinality for \(\mathfrak{t}\) would help in finding cardinality of \(\mathfrak{c}\).
    \item Proximal game
    \item Uniformities
    \item Homogeneity in Top group used to determine uniformity for player 1 in version of prox game for top groups
    \item FUF and 2FUF
    \item exercises:
    \begin{itemize}
        \item prove lemmas used in existence proof of towers.
        \item If \(\{A_{\alpha} \: : \: \alpha \subset \gamma \}\) is a tower and \(\{\alpha_{\xi} \: : \: \xi < \lambda\}\) is increasing and cofinal in \(\gamma\) then \(A_{\alpha_\xi} \: : \: \xi < \lambda\) is a tower.
        \item \(S_{\omega}\) is neither FUF or \(2FUF\) \checkmark
        \item Proximal game is equivalent to Gruenhage game. \textcolor{red}{Proximal space implies W \checkmark}
        \item Winning strategy in proximal game implies Frechet \textcolor{red}{I showed W and w space imply frechet}
    \end{itemize}
    \item note: send Paul changes to databse before pushing to github
\end{enumerate}

\subsection*{Wednesday May 28}
\begin{enumerate}
    \item More on Uniformities 
    \begin{itemize}
        \item Different metrics generate different Uniformities, for example discrete metric vs \(d(m, n) = |\frac{1}{m} - \frac{1}{n}|\) on \(\omega\), Uniformities from those metrics still generate the same (discrete) metric on \(\omega\)
        \item star refinements generalize triangle inequality, for example \(\{\ball(x, \frac{1}{2^{n + 1}}): x \in X\}\) star refines \(\{\ball(x, \frac{1}{2^n}): x \in X\}\) and gives triangle inequality
    \end{itemize}
    \item went over "proof" that proximal implies W, however Paul raised issue that player 2 in proximal game can use Hausdorffness to always pick points \(z\) that prompt player 1 in Proximal game to choose entourages \(U\) such that \(U[x] \cap U[p] = \nothing\), where \(p\) is where Gruenhage game being played. \textcolor{red}{this problem is fixed by starting Gruenhage game one round later than proximal game and player 2 pick p on turn 1 of proximal game}
    %\item paul says that in general there should be extra conditions on \(X\) in order that proximal implies W
    \item Exercises:
    \begin{itemize}
        \item If \(X\) semi-proximal then is FUF?  \textcolor{red}{turns out this isn't known (yet)}
        %\item \(X\) with one non-isolated point is such that Proximal implies W
        \item Is double arrow space Proximal?\checkmark
        \item Semi Proximal is \(\alpha_2\). \checkmark
    \end{itemize}
    \item Paul showed me proof that semi Proximal is Frechet
    \item Malykhin's problem: If \(G\) is a seperable Frehcet topological group, is it metrizable? Answer: Con(yes) and Con(No)
    \item \textbf{Paul's main question:} Can we strengthen hypotheses of Malykhin's problem to obtain answer that is not independent of ZFC?
\end{enumerate}
\subsection*{Thursday June 5}
\begin{enumerate}
    \item Discussed impliciations of conditions on filter \(\F\) when going from top space to top group 
    \item Questions:
    \begin{itemize}
        \item If \(\F\) is FUF, is \(\tau_{\F}\) semiproximal?
        \item If \(\tau_{\F}\) is semiproximal, is \(\F\) countable generated?
        \item Assuming CH, does there exist filters \(\F_0, \F_1\) both FUF neither countably generated where \(\tau_{\F_0}\) is semiproximal but \(\tau_{\F_1}\) isn't.
    \end{itemize}    
    \item Exercises: 
    \begin{itemize}
        \item Assume CH, construct FUF filter that is not countable generated. \checkmark
        \item What does it mean for a sequence in \((a_n)\) to converge to a point in \(G\), i.e., \((a_n)\to a\) iff what? 
        \item Consider a play of the Proximal game which yields sequences \((a_n)\) and \((V_{U_n})\). What does it mean for there to exist some \(b \in G\) such that \(b \in \bigcap_{n \in \omega}(a_n \Delta V_{U_n})\)? What does that look like?
    \end{itemize}
    \end{enumerate}

\subsection*{Thursday June 18}
\begin{enumerate}
    \item We talked about stars: how can we get a handle on the choices available to player 2?
    \item We looked at how sequences in \(\tau_{\F}\) behave, basically we just played with things trying to get a grasp at how things look
    \item More questions:
    \begin{itemize}
        \item In general, when is a topological group Proximal or Semi Proximal?
        \item If a top group is prox then is it first countable? (or contrapositive)
        \item In general what can we say about top groups and uniformities on them?
    \end{itemize}
    \item Check mathscinet for resources that could be helpful
\end{enumerate}



\section{Set Theory}
%\subsection{Ordinals and Cardinal}
%\begin{defn}
%    \(\aleph_1 = \omega_1\) (not definition but kind of?)
%\end{defn}
%\begin{thm}
%    \(|\P(\omega)| = |\bR|\).
%\end{thm}
%\begin{defn}[Continuum Hypothesis]
%    CH is the statement \(2^{\aleph_0} = \aleph_1\).
%\end{defn}
%CH is indepenet of ZFC: first shown by Gödel to not be inconsistent with ZFC and later shown by Cohen that its %negation is also consistent. Alternatively we can formulatie of CH as: \(\omega_1 = \P(\omega) = \mathfrak{c}\) %where \(\mathfrak{c} = |\bR|\).

\subsection{Some Interesting Cardinals}
\subsubsection{Subsets of $^\omega\omega$ and Almost Disjoint Familes}

\begin{defn}
    Let \(\A\) be an infinite family of infinite subsets of \(\omega\). The family \(\A\) is said to be an almost disjoint family (ADF) of subsets of \(\omega\) if \(A \cap B \) is finite for any \(A,B \in \mathcal{A}\) with \(A \ne B\). By Zorn's lemma we can assume \(\A\) lives inside of a maximal (w.r.t. inclusion) family that is also pairwise disjoint. Such a family is a maximal almost disjoint family (MADF).
\end{defn}
\begin{defn}
    Define an order \(\leq^{\ast}\) on \(^\omega\omega\) where \(f <^{\ast} g\) if \(f(n) < g(n)\) for all but finitely many \(n \in  \omega\)
\end{defn}
\begin{defn}
    Let \((A, \prec)\) be a linearly ordered set. We say that  \(B \subset A\) is unbounded in \(A\) if there does not exist \(y \in A\) such that \(x \prec y\) for all \(x \in B\). We say that \(C \subset A\) is cofinal (or dominating) in \(A\) if for all \(x \in A\) there exists \(z \in C\) such that \(x \prec x\).
\end{defn}
\begin{defn}
    \leavevmode
    \begin{itemize}
        \item \(\mathfrak{a} = \min\{|\A| \: : \: \A \text{ is infinte } \text{MADF}\}\)
        \item \(\mathfrak{b} = \min\{|B| \: : \: B \text{ is } \leq^{\ast}\text{-unbounded subset of }^\omega\omega\}\)
        \item \(\mathfrak{d} = \min\{|D| \: : \: D \text{ is } \leq^{\ast}\text{-cofinal in } ^\omega\omega
        \}\)
    \end{itemize}
    Each of \(\mathfrak{a}, \mathfrak{b}, \mathfrak{d}\) are at least \(\omega_1\) but at most \(\P(\omega)\).
\end{defn}
\subsubsection{Towers and Pseudo-Intersections}
\begin{defn}
    Define an order on \(\P(X)\) for some set \(X\). We say \(A \asubset B\) if \(B\) contains all but finitely many points of \(A\). If \(\F \subset \P(X)\) is a collection of (infinite) sets, then \(A\) is a pseudo-intersection of \(\F\) if \(A \asubset F\) for all \(F \in \F\).
\end{defn}
\begin{defn}
    We call \(\T \subset [\omega]^{\omega}\) a tower if \(\T\) is well ordered by \(\asubset\) and has no infinite intersection.
\end{defn}
\begin{lem}
    If \(\A = \{A_n \: : \: n \in \omega\} \subset [\omega]^{\omega}\) such that for all \(n \in \omega\)
    \[
        A_0 \asupset  A_1 \asupset \dots \asupset  A_n
    \]
    then \(\A\) has infintie pseudo-intersection.
\end{lem}
\begin{thm}
    There exists a tower.
\end{thm}
\begin{proof}
    Define \(\langle A_{\alpha} \: : \: \alpha < \mathfrak{c}\rangle\) as follows: 
    \begin{enumerate}
        \item \(A_{\alpha} = \omega \setminus n\) for \(\alpha < \omega\)
        \item If \(\alpha\) is a successor ordinal, define \(A_{\alpha}\) such that \(|A_{\alpha - 1}\setminus A_{\alpha}| =\aleph_0\)
        \item If \(\alpha\) is a limite ordinal then define \(A_{\alpha}\) such that \(A_{\alpha} \asubset A_{\beta}\) for all \(\beta < \alpha\)
    \end{enumerate}
    Suppose \(\gamma\) is given and \(\langle A_{\alpha} \: : \: \alpha < \gamma \rangle\) has been defined. If \(\gamma\) is a successor then define \(A_{\gamma}\) as in (2). Otherwise \(\gamma\) is a limite ordinal: if (3) isn't possible then \(\langle A_{\alpha} \: : \: \alpha < \gamma \rangle\) is a tower; otherwise define \(A_{\gamma}\) as in (3) and continue. 

    This process eventually terminates, otherwise we obtain \(\mathfrak{c}\) many distint subsets of \(\omega\), a contradiction. \textcolor{red}{TODO: elaborate}
\end{proof}
\begin{defn}
    \leavevmode
    \begin{itemize}
        \item \(\mathfrak{t} = \min\{|\T| \: : \: \T \text{ is a tower}\}\)
        \item \(\mathfrak{p} = \min\{|\P| \: : \: \P \subset [\omega]^\omega\text{ is a family with the SFIP that has no pseudo-intersection}\}\)
    \end{itemize}
\end{defn}
\section{Topology}
\subsection{Sequential and Fréchet Spaces}

\begin{defn}
    For a topological space \(X\) and any set \(A \subset X\), the \textit{sequential closure} of \(A\) is  
    \[
        \seqcl{A} := \left\{x \in X \: : \: \exists (x_n) \in A \left(\lim_{n\to \infty} x_n = x\right) \right\}.
    \]
    In general we can repeat this operation recursively \(\seqcl{\seqcl{\seqcl{A}}\dots}\) by which is meant the \textit{total sequential closure} of \(A\).
    
\end{defn}

\begin{fact}
    In general it takes at most \(\omega_1\) many iterations of the sequential closure to get a closed set. 
\end{fact}

\begin{defn}
    A space \(X\) is said to be Fréchet if \(\seqcl{A} = \overline{A}\) for all \(A \subset X\).
\end{defn}
\begin{exam}
    Let \(X = \omega_1 + 1\) with the order topology. \(X\) is not Fréchet, since any sequence \((x_n) \in \omega_1\) cannot converge to \(\infty\), as otherwise \(\omega_1 = \sup\{x_n \: : \: n \in \bN\}\), a contradiction.
\end{exam}
\begin{defn}
    A space \(X\) is sequential if any closed set \(A \subset X\) is equal to its total sequential closure.
\end{defn}
Then a space that is Fréchet is also sequential. The following example shows that the converse is not true.

 
\begin{exam}
    Let \(X^{\ast} = \omega \times (\omega + 1)\) be given the order topology and let \(X =  X^{\ast} \cup \{\infty\}\) where the neighboourhoods  of \(\infty\) are such that there exists \(p \in \omega\) such that \(|\{(m, n) \: : \: m > p, n \in \omega + 1\} \setminus U_{\infty}| < \aleph_0\). Then \(X\) is sequential but but not Fréchet. To see this, note that for all \(m \in \omega\) the sequence \(A_m = \{(m, n) \: : \: n \in \omega\}\) converges to \((m, \omega + 1)\) and moreover \(B = \{(m, \omega + 1) \: : \: m \in \omega\}\) is a sequence that converges to \(\infty\). Then \(A = \bigcup_{m \in \omega} A_m\) is such that \(\seqcl{\seqcl{A}} = X\), hence \(X\) is sequential. On the other hand there is no sequence in \(A\) that converges to \(\infty\). Suppose there were, say some \(\gamma \to \infty\). Then for all \(m \in \omega\), \(U_m = X \setminus \{(m, n)\: : \: n \in \omega + 1\}\) is a neighboourhood of \(\infty\) such that \(|\gamma \setminus U_m| < \aleph_0\). Hence \(\gamma\) has only finitely many terms belonging to each column. If \(\alpha_m = \max\{\gamma \cap \{(m, n) \: : \: n \in \omega\}\}\), then \(U = X \setminus \bigcup_{m \in \omega} \{(m ,n )\: : \:  n \leq \alpha_m\}\) is a neighbourhood of \(\infty\) disjoint from \(\gamma\), a contradiction. Hence \(X\) is not Fréchet. 
\end{exam} 
\begin{prop}
    If \(X\) is first countable then \(X\) is Fréchet.
\end{prop}
\begin{proof}
    Let \(A \subset X\) and let \(x \in \overline{A}\). Then \(x\) has a countable neighbourhood base \(N_x\) such that \(U \cap A \neq \nothing\) for all \(U \in N_x\). Enumerating the neighboourhoods of \(x\) as \(U_1, U_2, \dots\) then the sequence \((x_n)_{n \ge 1}\) where \(x_n \in U_n \cap A\) for each \(n \in \omega\) is such that \((x_n)_{n \ge 1}\) converges to \(x\).
\end{proof}

The following example shows that the converse is not true. 
\begin{exam}[Fréchet Fan]
    Let \(S_{\omega}\) be the quotient of \(\omega \times (\omega + 1)\) obtained by identifying all the points \(\{(m, \omega + 1) \: : \: m \in \omega\}\) as \(\infty^{\ast}\). More precisely \(S_{\omega}\) has the quotient topology induced by the map \(h: \omega \times (\omega + 1) \to S_{\omega}\) where \(h(x) = x\) for all \(x \in \omega \times \omega\) and \(h(x) = \infty^{\ast}\) for all \(x \in \omega \times \{\omega + 1\}\). Then \(S_{\omega}\) is Fréchet but not 1st countable. 
    
    To see that \(S_{\omega}\) is Fréchet, let \(A \subset S_{\omega}\) such that \(\infty \in \overline{A}\). \(A\) must meet at least one column of \(S_{\omega}\) in an infinite set, otherwise we could find a nbhd of \(\infty^{\ast}\) disjoint from \(A\). Then \(A\) restricted to that column will be a convergent sequence to \(\infty\).
    
    Now asssuming that \(S_{\omega}\) was countable, we would have a countable neighbourhood base at \(\infty^{\ast}\). For each \(k \in \omega\) let \(B_k = \bigcup_{m \in \omega}\{m\} \times (f_k(m), \infty^{\ast}]\) for some \(f_k: \omega \to \omega\) determining the startpoints of each interval. Suppose \(\B = \{B_k \: : \: k \in \omega\}\) is a base at \(\infty^{\ast}\), then let \(f^{\ast}: \omega \to \omega\) be defined by \(f^{\ast}(m) = f_m(m) + 1\) for all \(m \in \omega\). Letting \(B^{\ast} = \bigcup_{m \in \omega}\{m\} \times (f^{\ast}(m), \infty^{\ast}]\) then \(B^{\ast}\) is an open neighboourhood of \(\infty^{\ast}\) but it is clear by construction that \(B_k \not\subset B^{\ast}\) for all \(k \in \omega\). Hence \(\B\) cannot be a neighboourhood base and   \(S_{\omega}\) is not first countable.
\end{exam}

As the following example shows, the product of Fréchet spaces need not be Fréchet.
\begin{exam}
    Let \(X = (\omega + 1) \times S_\omega\), and consider the set \(A = \{(m, (m, n))\: : \:  m, n \in \omega\}\). If \(\infty^{\ast}\) is the identified point of \(S_{\omega}\), let \(\infty = \{\omega + 1\} \times \infty^{\ast}\). Then \(\infty \in \overline{A}\) but \(\infty \not \in \seqcl{A}\). The open neighboourhoods of \(\infty\) are are of the form \((\alpha, \omega + 1] \times\left(\bigcup_{m \in \omega}\{m\} \times (f(m), \infty^{\ast}]\right)\), which clearly always has non emtpy intersection with \(A\). Hence \(\infty \in \overline{A}\). To see that \(\infty \not \in \seqcl{A}\), suppose \(\gamma\) is a sequence in \(A\) that converges to \(\infty\). Since the sets \(U_k = (k, \omega + 1] \times\left(\bigcup_{m \in \omega}\{m\} \times (f(m), \infty^{\ast}]\right)\) are open neighbourhoods of \(\infty\) it must be the case that \(|\gamma \setminus U_k | < \aleph_0\) for all \(k \in \omega\). Thus  \(\gamma\cap\left(\{k\}\times\{k\}\times(1,\omega + 1]\right)\) is finite for every \(k\). Let \(h:\omega \to \omega\) be defined by \(h(k) = \max\{\pi_3(\gamma\cap\left(\{k\}\times\{k\}\times(1,\infty^{\ast}]\right)\} + 1\) for \(k \in \omega\). Pictorially, \(h\) is picking the point on each spine beyond which no elements of \(\gamma\) exist. Thus 
    \[
    W = (1, \omega + 1] \times\left(\bigcup_{n \in \omega}\{n\}\times(h(n), \infty^{\ast}]\right)
    \] 
    %\[
    %W = \left(\bigcup_{m \in \omega} \bigcup_{n \in \omega}\{m\}\times\{n\}\times(h(n), \infty^{\ast}]\right)\cup\left(\{\omega + 1\}\times S_{\omega}\right)
    %\] 
    is an open neighbourhood of \(\infty\) which by construction is disjoint from \(\gamma\). Hence \(\gamma\) cannot converge to \(\infty\) showing that \(X\) is not Fréchet.
\end{exam}

\subsection{\(\alpha_i\) notions of convergence}

\begin{defn}
    Let \(X\) be a topological space and \(\xi\) be a countable family of sequences converging to a point \(x \in X\). We say that \(x\) is an \(\alpha_i\) point for \(i = 1, 2, 3, 4\) if there exists a sequence \(B\) such that 
    \begin{itemize}
        \item \(\alpha_1\):  \(|A \setminus B| < \aleph_0\) for every \(A \in \xi\);
        \item \(\alpha_2\): \(|A \cap B| = \aleph_0\) for every \(A \in \xi\);
        \item \(\alpha_3\): \(|A \cap B| = \aleph_0\) for infinitely many \(A \in \xi\);
        \item \(\alpha_4\): \(A \cap B \neq \nothing\) for infinitely many \(A \in \xi\).
    \end{itemize}
    Then \(X\) is an \(\alpha_i\) space if every \(x \in X\) is an \(\alpha_i\) point. Note that if a space is \(\alpha_i\) then it is \(\alpha_{i + 1}\) for \(i =1, 2, 3\).
\end{defn}

\begin{exam}
    \(S_{\omega}\) is not even \(\alpha_4\). For each \(m \in \omega\) let \(A_m = \{m\} \times (1, \infty)\). Then \(\xi = \{A_m \: : \: m \in \omega\}\) is a countable collection of sequences converging to \(\infty\). Suppose \(B\) is a sequence that converges to \(\infty\) such that \(A\cap B \neq \nothing\) for infinitely many \(A \in \xi\). In particular let \(\alpha \leq \omega\) be such that \(A_i \cap B \neq \nothing\) for all \(i \in \alpha\) and let \(f:\alpha \to \omega\) be defined by \(f(k) \in B \cap A_k\) for all \(k \in \alpha\). Then 
    \[
    U = \left(\bigcup_{k \in \alpha} \{k\} \times (f(k) + 1, \infty]\right)\times \left(\bigcup_{m \in \omega\setminus\alpha} \{m\} \times (1, \infty]\right)\] 
    is such that \(|B \cap U^C| = \aleph_0\), hence \(B\) does not converge to \(\infty\). 

\end{exam}

\begin{exercise}{\label{var_alpha2}}
    X is \(\alpha_2\) iff whenever \(\xi\) is a countable collection of sequences converging to \(x\) there exists \(B \to x\) such \(A \cap B \neq \nothing\) for all many \(A \in \xi\).
\end{exercise}
\begin{soln}
    The forward direction is obvious. Conversely, let \(\xi =\{A_1, A_2, \dots\}\) be a countable collection of sequences converging to \(x\). For every \(n\in \omega\) let \(\{a_{nm} \: : \: m \in \omega\}\) enumerate the elements of \(A_n\) and define \(A_{nm} = A_n \setminus \{a_{n1}, a_{n2}, \dots, a_{n,m-1}\}\) for each \(n,m \in \omega\). Then the \(A_{nm}\) still convege to \(x\) and  \(\A = \{A_{nm} \: : \: n,m \in \omega \}\) is a sheaf at \(x\). By hypothesis there exists a \(B\) converging to \(x\) that meets each \(A_{nm}\), and thus meets each \(A_n\) in an infinite set.
\end{soln}

\subsection{\(\Psi\)-like spaces}

\begin{defn}
    Let \(D\) be a countable set and let \(\A \subset \P(D)\) be an ADF. Define a topology on \(X = D \cup \A\) such that the points of \(D\) are isolated and \(D\) is dense in \(X\). Then for every \(A \in \A\) attach a point \(z_A\) where the nbhds of \(z_A\) consist of all sets of the form \(B(A, F) = \{z_A\} \cup (A \setminus F)\) where \(F \subset A\) is finite.
\end{defn}

\begin{prop}
    Let \(X\) be as above. Then \(X\) is Hausdorff, first countable and locally compact. If \(X\) is constructed as above but \(\A\) is MADF then \(X\) is also pseudo-compact (i.e. all continuous real functions on \(X\) are bounded).
\end{prop}

\begin{proof}
    To check Hausdorffness it suffices to show \(Z = X\setminus D\) is Hausdorff. Let \(z_1, z_2 \in Z\). Then \(z_1, z_2\) correspond to sets \(A_1, A_2\) such that \(A_1 \cap A_2 = I\) is finite. Then \(B(A_1, I)\) and \(B(A_2, I)\) are disjoint open sets containing \(z_1, z_2\) respectively. 

    To see that \(X\) is locally compact, note that \(\{x\}\) is a compact neighbourhood base for all \(x \in D\). Local compactness of \(Z\) follows from the fact that by design each \(z_A \in Z\) is the limit of \(A\) viewed as a sequence. In other words, whenever \(B(A_z, F)\) is a basic nbhd of \(z_A\), any open cover will have a smaller nbhd \(B(A_z, F')\) that omits only finitely many points.
\end{proof}

\begin{prop}
    The one point compactification of a \(\Psi\)-like space of cardinality \(<\mathfrak{a}\) is Fréchet. 
\end{prop}

\subsection{Constructing Examples of \(\alpha_i\) Spaces}
\subsubsection{Squares of \(\alpha_i\) Spaces} 
\begin{exam}
    If \(X, Y\) are \(\alpha_1\), then so is \(X \times Y\). Let \(\xi = \langle \xi_n \: : \: n \in \omega \rangle \) be a  sheaf at some \(x\times y \in X \times Y\). Then \(\pi_i(\xi) := \langle \pi_i(\xi_n) \: : \: n \in \omega\rangle\) is a sheaf at \(\pi_i(x\times y)\) in \(X, Y\) respectively for \(i = 1, 2\). Hence there exists \(\alpha_1\) sequences \(B_x, B_y\) for \(\pi_i(\xi)\). Letting \(M_n = \max\{k \in \omega \: : \: x_{n,k} \not \in \pi_1(\xi_n) \vee y_{n,k} \not \in \pi_2(\xi_n)\} \) for each \(n \in \omega\), the sequence \(B = \bigcup_{n \in \omega}B_x \cap (\xi_n\setminus {x_{n1}, \dots, x_{n, M_n}}) \times B_y \cap (\xi_n\setminus {y_{n1}, \dots, y_{n, M_n}})\) is as desired.
\end{exam}

\subsubsection{Misc \(\alpha_i\) Spaces}

\begin{exam}
     Let \(D = \omega\times\omega\), let \(\F = \{f_{\alpha} \: : \: \alpha < \mathfrak{b}\} \subset \, ^\omega\omega\) be \(<^{\ast}\)-well-ordered and \(<^{\ast}\)-unbounded, and let \(\C = \{C_n = \{n\} \times \omega \: : \: n \in \omega\}\). Construct a \(\Psi\)-like from \(X = \F \cup \C\), where we attach compactification points \(\{n\} \times (\omega+1)\) for each \(C_n\) and compactification points \(p_{\alpha}\) for each \(f_{\alpha}\). Then the resulting one point compactification, \(X + \infty\), is \(\alpha_2\) and Fréchet but not \(\alpha_1\). \(X + \infty\) becomes \(\alpha_1\) if \(\F\) is cofinal and well ordered. 
\end{exam}

\subsection{Frechet-Urysohn for Finite Sets}

\begin{defn}
    A \(\pi\)-network at a point \(x \in X\) is a collection \(\F \subset \P(X)\) such that for all open nbhds \(U_x\) there is \(F \in \F\) such that \(F \subset U_x\).
\end{defn}
\begin{defn}
    We say that an infinite family of sets  \(\langle F_n \subset \P(X) \: : \: n \in \omega\rangle\) converges to a point \(x \in X\) if \(|\{F_n \: : \: F_n \not \subset U_x\}| < \aleph_0\) for every open nbhd \(U_x\). 
\end{defn}
\begin{defn}
    \(X\) is Fréchet-Urysohn Finite (FUF) if for all \(x \in X\) whenever \(\F \subset [X]^{<\omega}\) is a \(\pi\)-network at \(x\) there exists \(\langle F_n \: : \: n \in \omega\rangle \subset \F\) such that the \(F_n\) converge to \(x\). We say that \(X\) is \(n\)FUF if the condition holds for all \(\pi\)-networks \(\F \subset [X]^n\) and we say that \(X\) is boundedly FUF if \(X\) is \(n\)FUF for all \(n \in \omega\).
\end{defn}

\begin{exam}
    \(S_{\omega}\) is not 2FUF (and hence not FUF). Let \(F_m^n = \{(1, m), (m + 1, n)\}\) and let \(\F = \{F_m^n \: : \: m,n \in \omega\}\). Then \(\F\) is a \(\pi\)-network at \(\infty\) with no convergent subcollection. 
    
    To see that \(\F\) is a \(\pi\)-network, let \(U\) be an open nbhd of \(\infty\). Then \(U = \bigcup_{k \in \omega} \{k\} \times (f(k), \infty]\) for some \(f \in \, ^{\omega}\omega\). Letting \(m = f(1) + 1\) and \(n = f(m) + 1\), then \(F_m^n \subset U\).

    Suppose \(\G \subset \F\) converges to \(\infty\). Then since \(V_n = S_{\omega}\setminus \left(\{1\} \times (1, n]\right)\) is an open nbhd of \(\infty\) for each \(n \in \omega\), we see that the sets \(\{G \in \G \: : \: G \not \subset V_n\}\) are each finite. In particular, this shows that there are only finitely many \(G \in \G\) that meet each column, hence for each \(n \in \omega\) there exists \(a_n \in \omega\) such that for all \(b \geq a_n\), \(F_n^b \not \in \G\). Letting \(g:\omega \to \omega\) be defined by \(g(n) = a_n\) we obtain the open nbhd \(W = \bigcup_{n \in \omega} \{n\} \times (g(k), \infty]\) which is such that \(G \not \in W\) for all \(G \in \G\), i.e. \(\{G \in \G \: : \: G \not \subset W\}\) is infinite so that \(\G\) does not converge to \(\infty\).
\end{exam}
\begin{prop}
    If \(X\) is first countable then \(X\) is FUF.
\end{prop}
\begin{proof}
    Suppose \(X\) is first countable, let \(x \in X\), and let \(\F\) be a \(\pi\)-network at \(x\). Let \(\U = \{U_0, U_1, \dots\}\) be a nbhd base at \(x\) and WLOG assume  that \(U_0 \supset U_1 \supset \dots\) (since otherwise we could just take \(U_0, U_0\cap U_1, \dots\)). Then for each \(n \in \omega\) there is a \(\F \ni F_n \subset U_n\) and the \(F_n\) converge to \(x\).
\end{proof}
\begin{defn}
    \(\A \subset \P(X)\) is a network if for all \(x \in X\) and for all nhds \(U \ni x\) there is \(A \in \A\) such that \(x \in A \subset U\). If \(\F\) is a filter, then we might also refer to a network \(\A\) for \(\F\) where every \(F \in \F\) contains some \(A \in \A\).
\end{defn}
\begin{rem}
    A network is a generalization of a base since we no longer require the elements to be open sets.
\end{rem}
\begin{defn}[FUF filters]
    A filter \(\F\) on \(\omega\) is FUF if whenever \(A \subset [\omega]^{<\omega}\) is a network for \(\F\) then there exists \(B \subset A\) such that \(B\) converges wrt to \(\F\).
\end{defn}
\begin{lem}
    Let \(\F\) be a filter on \([\omega]^{\omega}\). If \(\F\) is countably generated then \(\F\) has pseudo-intersection.
\end{lem}
%\begin{proof}
%    Suppose \(\{F_n : n < \omega\}\) generates \(\F\). Let \%(a_n \in (\bigcap_{i < n}F_i)\setminus\{a_i:i<n\}\). Then \%(\{a_n:n<\omega\} \asubset F\) for all \(F \in \F\).
%\end{proof}

\begin{prop}
    Assume CH. Then there exists a FUF filter \(\F\) that is not countably generated.
\end{prop}
\begin{proof}
    Starting with \(\F_0\) the cofinite filter on \(\omega\), we will recursively construct a filter by adding a new set to our filter at each stage such that the FUF condition is preserved. We assume that \(\P([\omega]^{<\omega}) = \{A_n : n < \omega_1\}\) and will construct our filter in \(\omega_1\) many steps.

    On step 0, take \(A_0 \subset [\omega]^{<\omega}\). If it's a network for \(\F_0\) then we find \(B_0 \subset A_0\) such that \(B_0 \to \F_0\), which exists since \(\omega\) is first countable. Otherwise, if \(A_0\) isn't a network let \(B_0\) be the empty sequence and proceed to the next step. 

    On step \(\alpha\), let \(\F_{\alpha}^{\ast} = \F_0\cup \left(\bigcup_{i < \alpha}\{X_i\}\right)\) and let \(\F_{\alpha} = F_{\alpha}^{\ast}\cup\{X_{\alpha}\}\) where \(X_{\alpha}\) will be defined in such a way as to preserve any previous convergent sequences, which we denote \(\{B_{\beta}:\beta < \alpha\}\). Note that \(\F_{\alpha}^{\ast}\) is countably generated, as \(\F_0\) is countably generated and we have progressed only countably many steps. So we have \(F_0 \supset F_1 \supset \dots\) which generate \(\F_{\alpha}^{\ast}\). Let \(X_{\alpha} = \bigcup_{i < \alpha}F_i\cap (\cup B_i)\). Since each \(F_i\) contains a tail of every \(B_{\beta}\), we have that \(X_{\alpha} \asupset B_{\beta}\) for all \(\beta\). On the other hand, none of the \(F_n\) generate \(X_{\alpha}\) since \(F_1 \cap B_1 \not\supset F_k\) for all \(k \geq 1\) and in general \(F_n \cap B_n \not \supset F_k\) for all \(n \in \omega, k \geq n\). Hence \(X_{\alpha}\not\in \F_{\alpha}^{\ast}\) and \(\F_{\alpha}\) is as desired.

    We continue step \(\alpha\) by taking \(A_{\alpha} \subset [\omega]^{\omega}\). If it's a network for \(\F_{\alpha}\), find another sequence \(B_\alpha \subset A_{\alpha}\) that converges with respect to \(\F_{\alpha}\), otherwise if \(A_{\alpha}\) isn't a network let \(B_{\alpha}\) be the empty sequence proceed to the next step.

    Let \(\F\) be the filter obtained after recursively iterating through all \(A_n \subset [\omega]^{<\omega}\), i.e. \(\F = \F_0 \cup \left(\bigcup_{\alpha < \omega_1}\{X_\alpha\}\right)\). By construction \(\F\) is FUF and to see that its uncountably generated let \(\G \subset \F\) be a generating set. Then for some \(\alpha < \omega_1\) there's \(\G_{\alpha} \subset \G\) which generates \(\F_{\alpha}^{\ast}\). But \(\F_{\alpha}^{\ast}\) was also generated by \(F_0 \supset F_1 \supset \dots\) and \(X_{\alpha}\) was such that \(X_{\alpha}\not\supset F_n\) for all \(n \in \omega\). Since the \(F_n\) would necessarily also generate \(\G_{\alpha}\) this implies that \(X_{\alpha} \not \supset G_n\) for all \(n \in \omega\). Hence \(\G_{\alpha}\) generates \(\F_{\alpha}^{\ast}\) but not \(\F_{\alpha}\). Furthermore, \(\G_{\alpha}\) cannot generate \(\F_{\eta}\) either, for \(\alpha < \eta < \omega_1\), which follows from the definition of the \(X_{\eta}\). Thus as \(\F\) is obtained from \(F_{\alpha}\) by adding uncountably many new sets, each of which not generated by \(\G_{\alpha}\), it must be the case that \(\G\) is uncountable. 
\end{proof}

\subsection{Uniformities}
    In general uniformities provide an alternate way to generalize metric spcaes. Topological spaces generalize metric spaces while preserving the notion of continuous functions whereas uniform spaces preserve  uniformly continuous functions. Given a set \(X\) there are two ways to contruct a uniformity \(\D\) on \(X\). 
    
\subsubsection{Diagonal Uniformities}
    A diagonal uniformity on a set \(X\) is a principal filter \(\D\) on the square \(X \times X\) generated by the diagonal satisfying 
    \begin{enumerate}
        \item For all \(D \in \D\) there exists \(E \circ E \subset D\) for some \(E \in \D\)
        \item \(D^{-1} \in \D\) for all \(D \in \D\)
    \end{enumerate}
    where \(E \circ E  = \{(x, y) \in E \times E: \exists z \in E\,((x,z) \in E \wedge (z, y) \in E)\}\) and \(D^{-1} = \{(y, x): (x, y) \in D\}\). Note that \(E\circ E\) will often be abreviated \(E^2\) and we may at times go so far as to use \(E^4 = (E^2)^2 = E\circ E\circ E\circ E\). Also, whenever \(D \in \D\) is such that \(D = D^{-1}\) we will say that \(D\) is symmetric. A diagonal uniformity can be generated by filterbase consisting of symmetric sets \(D\) each containing the diagonal. The pair \((X, \D)\) will be called a uniform space.

    If \((X, \D)\) is a uniform space then for any \(x \in X\) and \(D \in \D\) we say that \(D[x] = \{y \in X:(x, y) \in D\}\) is the section of \(D\) at \(x\). More generally, we can also talk about \(D[A]=\{y \in X:(x, y) \in D, x \in A\}\). Although the former is used quite frequantily because the collection \(\tau_{\D} = \{D[x]:x \in X, D \in \D\}\) is a base for a topology on \(X\), and the most natural way to pass from uniform to topological spaces.
\subsubsection{Uniform Covers}

\subsection{Topological Groups}
\begin{defn}
    A topological group \(G\) is a group whose underlying set has a topology such that 
    \begin{itemize}
        \item the group operation is continuous;
        \item the map that sends \(x \mapsto x^{-1}\) is continuous.
    \end{itemize}
    For sets \(A, B \subset G\) and any point \(g \in G\) we denote \(AB = \{a\cdot b \: : \: a \in A, b \in B\}\) and \(gA = \{g\cdot a \: : \: a \in A\}\).
\end{defn}
 
%Topological groups provide a natural way of defining uniform covers on \(G\). Fix \(V\) an open set containing \(e\) and let \(\U = \{gV \: : \: g \in G\}\). Then \(G\) is a uniform cover. (check this)

We will often construct topological groups from topological space. Take \(X = \{\infty\} \cup \omega\) with the points of \(\omega\) isolated and let \(\F\) a filter on \(\omega\). Then \(G = [\omega]^{<\omega}\) with the operation symmetric difference \(\Delta: G \times G \to G\) defined by \(a\Delta b = a\setminus b \cup b\setminus a\) and \(\nothing\) the identity will be our group. For each \(U \in \F\) set \(V_U \subset [\omega]^{<\omega}\) where \(V_U = \{a \in [\omega]^{<\omega}: a \subset U\}\). Then \(\V = \{V_U : U \in \F\}\) is a nbhd base at \(\nothing\) and \(\tau_{\F} = \{a\Delta V_U : a \in [\omega]^{<\omega}, U 
\in \F\}\) is a base for a topology on \(G\).

\begin{prop}
    If \(\F\) is FUF then \(\tau_{\F}\) is Frechet.
\end{prop}
\begin{proof}
    Let \(\F\) be FUF and let \(A \subset \tau_{\F}\) be such that \(x \in \overline{A}\). WLOG assume that \(x = \nothing\). Then \(\A = \{a \in A \cap [F]^{\leq\omega}: F \in \F\) is a network for \(\F\) and hence there is a \(B \subset \A\) such that \(B \to \F\), i.e., \(B \to x\).

    Conversely, let \(\A\) be a network for \(\F\) and let \([\A]^{<\omega} = \{[a]^{<\omega}:a \in \A\}\). Then \(\nothing \in \overline{[\A]^{<\omega}}\) since there is \(a \subset F\) implies \([a]^{<\omega} \subset [F]^{<\omega}\) hence \([\A]^{<\omega}\cap [F]^{<\omega}\neq \nothing\) for all \(F \in \F\). Thus there is a \(B \subset [\A]^{<\omega} \) that converges to \(\nothing\) and hence \(B \to \F\).
\end{proof}
\begin{prop}
    \(\F\) is countably generated \(\iff\) \(\tau_{\F}\) is secound countable \(\iff\) \(\tau_{\F}\) is metrizable.
\end{prop}

\begin{exam}
    If \(\F\) is the cofinite filter on \(\omega\) then in the group \(\tau_{\F}\) the sequence defined by \(x_n^m = \{n, n + 1, \dots, n + m\}\), for any \(m < \omega\),  converges to \(\nothing\). Letting \(m = 1\) and using the homogeneity of \(\tau_{\F}\) we obtain a simple recipe to construct sequences converging to any point \(p \in [\omega]^{<\omega}\); just take \(y_n = x_n^1 \Delta p\) for all \(n \in \omega\).
\end{exam}

\section{Topological Games}
%In this chpater we assume \(X\) to always be Hausdorff.
\subsection{Gruenhage Game}

Let \(x \in X\) be a point chosen before the start of game. On turn zero, player 1 chooses an open nbhd \(U_0\) of \(x\) and player 2 chooses a point \(x_0 \in U_0\); on the \(n\)'th turn player one chooses an open nbhd \(U_n\) of \(x\) and player 2 chooses a point \(x_n \in U_n\). Player 1 wins if the sequence of points \(\{x_n \: : \: n \in \omega\}\) converges to \(x\).

Formally, we can describe a strategy for player 1 as a mapping \(\sigma: [X]^{<\omega} \to \O_x\) from finite sequences in \(X\) to the collection of open sets containing \(x\) where \(\sigma(\nothing) = X\). In particular we say that \(\langle x_n \: : \: n \in \omega\rangle\) is a \(\sigma\)-sequence provided \(x_{n + 1} \in \sigma(\{x_1, \dots, x_n\})\) for all \(n\). We say that \(\sigma\) is a winning strategy if every \(\sigma\)-sequence converges to \(x\).

A strategy for play 2 is a mapping \(\tau: \S(x) \times \O_x \to X\) such that \(\tau(F, U) \in U\) for all \(F \in \S(x)\) and \(U \in \O_x\).

\begin{defn}
    We say that \(X\) is a W-space if there is a winning strategy for player 1 at every point \(x \in X\).  We say that \(X\) is a w-space if for every strategy for player 2 there exists a counterstrategy for player 1 that wins.
\end{defn}
%\begin{prop}
%    Every W-space is \(\alpha_2\) and Fréchet.
%\end{prop}
%\begin{proof}
%    Let \(X\) be a W-space. Let \(A \subset X\) and \(x \in %\overline{A}\). If \(\sigma_x: [X]^{<\omega} \to \O_x\) is a %winning strategy at \(x\) then the sequence \(\langle x_n \: %: \: n \in \omega\rangle\) where \(x_{n + 1} \in \sigma(\%{x_1, \dots, x_{n + 1}\})\cap A\) for all \(n\) is a \%(\sigma_x\)-sequence and therefore converges to \(x\). %Supposing \(\xi\) to be a sheaf at some \(y \in X\) and \%(\sigma_y\) a winning strategy at \(y\), we define a %sequence as follows: \(y_1 \in \xi_1\) and \(y_{n + 1} \in %\sigma_y(\{y_1, \dots, y_n\}) \cap \xi_{n + 1}\). Then \%(\langle y_n \: : \: n \in \omega\rangle\) is a \(\sigma_y\)%-sequence and converges to \(y\); moreover it has non-empty %intersection with each member of \(\xi\) which by \ref%{var_alpha2} this is equivalent to the \(\alpha_2\) property.
%\end{proof}

\begin{defn}
    Let \(x \in X\) and let \(\tau: \S(X) \times \O_x(X) \to X\) be a strategy for player 2 at \(x\). Say that \(\F_{\tau} = \langle F_n  \: : \: n \in \omega\rangle\) is a \(\tau\)-chain if for all \(n \in \omega\) we have \(F_n \subset F_{n + 1}\) and \(F_{n + 1} = F_n \cup \{\tau(F_n, U)\}\) for some \(U \in \O_x\).
\end{defn}

\begin{prop}
    \(X\) is a w-space iff it is  \(\alpha_2\) and Fréchet.
\end{prop}
\begin{proof}
    The forward direction is similar to above result, with the difference that we define a strategy \(\tau\) for player 2 such that \(\tau(F, U_x) \in U_x \cap A\) on the one hand, and \(\tau(F, U_y) \in U_y \cap \xi_{|F|}\) on the other hand; then since \(X\) is a w-space, we obtain a counter strategy that wins for player 1, i.e., the desired convergent sequence.

    Now suppose \(X\) is \(\alpha_2\) and Fréchet and let \(\tau: \S(X) \times \O_x(X) \to X\) be a strategy for player 2 at \(x\). For each \(F \in \S(X)\) let \(\tau_F = \{\tau(F, U) \: : \: U \in \O_x\}\) and note that \(x \in \overline{\tau_F}\) for all \(F\). By Fréchetness, each \(\tau_F\) contains an \(A_F\) that converges to \(x\) and \(\xi = \langle A_F \: : \: F \in \S(X)\)  is a sheaf at \(x\). Let \(B\) be a sequence that converges to \(x\) which meets each \(A_F\) and let \(\sigma: [X]^{< \omega} \to \O_x\) be a counterstrategy for player 1 such that \(\tau(F, \sigma(F)) \in B \cap A_F\) for all \(F \in [X]^{< \omega}\). Then \(\sigma\) wins for player 1. \textcolor{red}{This assumes $X$ is countable.}

    Suppose \(X\) is \(\alpha_2\) and Fréchet and let \(x \in X\) be a point. If \(\tau: \S(X) \times \O_x(X) \to X\) is a strategy for player 2 at \(x\), then for each \(F \in \S(X)\) let \(\tau_F = \{\tau(F, U) \: : \: U \in \O_x\}\) and note that \(x \in \overline{\tau_F}\), which by Fréchetness of \(X\), implies the existence of a \(A_F \subset \tau_F\) that converges to \(x\)  for each \(F \in \S(X)\). Let \(\mathfrak{F}\) be a collection of \(\tau\)-chains such that for each \(y \in X\) there exists exactlly one \(\F_{\tau_y} \in \mathfrak{F}\) such that \(\{y\} \in \F_{\tau_y}\) (AC). Observe that to each \(\F_{\tau_y} \in\mathfrak{F}\) there corresponds a countable sheaf \(\xi_y = \langle A_F \: : \: F \in \F_{\tau_y}\rangle\) and as \(X\) is \(\alpha_2\), we therefore obtain for each \(\F_{\tau_y}\) a sequence \(B_y\) that meets each \(A \in\xi_y\) and converges to \(x\). 
    
    Let \(\sigma: \S(X) \to \O_x\) be a counter strategy for player 1 defined recursively such that \(\tau(\{y\}, \sigma(\{y\})) \in B_y \cap A_{\{y\}}\) and \(\tau(F, \sigma(F)) \in B_y\cap A_F\) if \(F \in \F_{\tau_y}\). To see that \(\sigma\) wins for player 1, let \(\langle x_n \: : \: n \in \omega\rangle\) be a \(\sigma\)-sequence. Letting \(F_n = \{x_1, \dots, x_n\}\) for all \(n \in \omega\) we observe that \(F_1 = \{x_1\}\), \(x_{n + 1} \in \sigma(F_n)\), and \(x_{n+1} = \tau(F_n, \sigma(F_n))\). Thus the collection \(\langle F_n \: : \: n \in \omega\rangle\) forms the \(\tau\)-chain \(\F_{\tau_{x_1}}\) which implies \(\langle x_n \rangle = B_{x_1}\), hence \(\langle x_n \: : \: n \in \omega\rangle\) converges to \(x\).
    %Suppose \(X\) is \(\alpha_2\) and Fréchet and let \%(\tau: [X]^{<\omega} \times \O_x(X) \to X\) be a %strategy  at \(x \in X\) for player 2 . For each \(F \in %[X]^{<\omega}\) let \(\tau_F = \{\tau(F, U) \: : \: U %\in \O_x\}\) and note that \(x \in \overline{\tau_F}\) %for all \(F\). By Fréchetness, each \(\tau_F\) contains %an \(A_F\) that converges to \(x\).  Let \(\mathfrak{F}%\) be the set of all \(\tau\)-chains in \(X\) and note %that for each \(\F_\tau \in \mathfrak{F}\) there exists %a corresponding countable sheaf \(\xi_{\F_\tau} = %\langle A_F \: : \: F \in \F_\tau\rangle\) as well as a %sequence \(B_{\F_\tau}\) that meets each \(A_F \in \xi_%{\F_\tau}\) and converges to \(x\). For each \(x \in X\) %pick \(\F_{\tau_x} \in \mathfrak{F}\) such that \(\{x\} %\in \F_{\tau_x}\). Define \(\sigma: [X] \to \O_x\)  such %that \(\tau(F, \sigma(F)) \in B_{\F_{\tau}} \cap A_F\) %for all \(F \in [X]^1\) where \(\F_{\tau}\) is a \(\tau\)%-chain containing \(F\) and let \(\sigma_{n + 1}: [X]^{n %+ 1} \to \O_x\) be such that \(\tau(F, \sigma_n(F)) \in %\). To see that \(\sigma\) wins for player 1, let \%(\langle x_n \: : \: n \in \omega\rangle\) be a \%(\sigma\)-sequence. Letting \(F_n = \{x_1, \dots, x_n\}%\) for all \(n \in \omega\) we observe on the one hand %that \(x_{n + 1} \in \sigma(F_n)\), since \(\langle %x_n\rangle\) is a \(\sigma\)-sequence, and on the other %hand, that \(x_{n+1} = \tau(F_n, \sigma(F_n))\). Thus %the collection \(\langle F_n \: : \: n \in %\omega\rangle\) forms a \(\tau\)-chain \(\F_{\tau}\). %Thus \(\langle x_n \rangle \subset B_{\F_{\tau}}\) and %therefore converges to \(x\).
\end{proof}

\subsection{Proximal Game}

This game is due to Bell and is described as follows. Let \((X, \D)\) be a uniform space. On turn zero, player 1 chooses an entourage \(D_0 \in \D\) and player 2 chooses a point \(x_0 \in X\). On the next turn, player 1 chooses \(D_1 \in \D\) with \(D_1 \subset D_0\) and player 2 chooses \(x_1 \in D_0[x_0]\). The game continues and in general on the \(n\)'th turn player 1 chooses \(D_{n + 1} \in \D\) with \(D_{n + 1} \subset D_n\) and player 2 chooses \(x_{n + 1} \in D_n[x_n]\). Player 1 wins if either: 

\begin{enumerate}
    \item there exists \(x \in X\) such that the \(\langle x_n \: : \: n \in \omega\rangle\) converge to \(x\);
    \item \(\bigcap_{n \in \omega}D_n[x_n] = \nothing\).
\end{enumerate}
\begin{defn}
    A strategy for player 1 in the Proximal game is a mapping \(\rho: \S(X) \to \D\) such that \(x_{n + 1} \in \rho(F)[x_n]\) for all \(n \in \omega\) and \(\rho(F) \supset \rho(G)\) if \(G \subset F\). Note that \(F \in \S(X)\) represents player 2's first \(n\) choices if \(|F| = n\) and \(\rho(F)\) represents player 1's choice on turn \(n + 1\). We will denote \(\rho(\nothing)\) as \(X \times X\). 
\end{defn}
\begin{defn}
    We say that \(X\) is Proximal if player 1 has a strategy \(\rho\) such that every \(\rho\)-sequence converges. We say that \(X\) is Semi-Proximal if for every strategy for player 2, player 1 has a counter strategy that wins.
\end{defn}
\begin{rem}
    Paul explained the game to me in a slightly different way. \(X\) is assumed to be compact. On turn one player 1 begins by playing a finite open cover (FOC) \(\U_1\) and player 2 continues by picking a point \(x_1 \in X\). On the second turn player 1 chooses a FOC \(\U_2\) such that \(\U_2 \prec \U_1\) and player 2 picks a point \(x_2 \in \st(x_1, \U_2)\). Then player 1 wins if either \(\bigcap_{n \in \omega}\st(x_n, \U_n) = \nothing\) or the \(\langle x_n \rangle\) converge to some point. 
\end{rem}
\begin{exam}
    The double arrow space is \(X = [0, 1]\times\{0, 1\}\) with the order topology; it is not Semi-Proximal. As \(X\) is compact, there exists a unique uniformity that generates the topology. Player 1 plays FOCs of the form \(\U = \{(a, b] \times \{0\} \cup [a, b) \times \{1\} \: : \: a,b \in [0, 1]\}\) and player 2 picks points from alternating rows every turn. Player 2's strategy will always be viable, since the star of any point wrt to any FOC will always intersect both rows.
\end{exam}
\begin{prop}
    If \(X\) is semi-proximal then \(X\) is \(\alpha_2\) Fréchet. 
\end{prop}
\begin{proof}
    Suppose \(X\) is not Fréchet. Then there exists some \(A \subset X\) with \(p \in \overline{A}\) such that no sequence in \(A\) converges to \(p\). Then player 2 can addopt a strategy as follows. On turn one, player 1 picks \(U_1\) and player 2 picks \(x_1 = p\). On turn 2, player 1 picks \(U_2\) and player 2 picks \(x_2 \in U_2\cap A[p] \), which by symmetry allows player 2 to pick \(p \in U_2\cap A[x_2]\) on the tird turn. Thus is general, player 2 picks points in \(A\) such that \(p\) always remains a valid play on the next turn, and continues to pick \(p\) ever other turn. By symmetry, \(p \in \bigcap U_i[x_i]\), and by assumption, the sequence chosen by player 2 cannot converge, hence player 1 loses so that \(X\) is not semi-proximal.

    Similarly, if \(X\) is not \(\alpha_2\), then we can find can find a point \(p\) and a sheaf \(\xi\) at \(p\) such that any sequence that meets each \(\xi_i\) doesn't converge to \(p\). As before, player 2 picks \(p\) on turn one, but on turn 2, picks \(x_2 \in U_2 cap \xi_1\)and in general picks \(x_{2n} \in U_{2n}\cap\xi_{n}\). 
\end{proof}
%A particular flavour of this game can be played in the setting of topological groups where the uniformity is defined in the natural way.

\begin{prop}
    If player 1 (player 2) wins (loses) in the proximal game, %then player 1 (player 2) wins (loses) in the Gruenhage game.
\end{prop}
\begin{proof}
    Let \((X, \D)\) be a uniform spaces and let \(\rho\) be a winning strategy for player 1 in the proximal game. Let \(\tau\) be a topology on \(X\) generated by the basis \(\B = \{D[x] \: : \: D \in \D, x \in X\}\), let \(x \in X\), and let \(\sigma: \S(X) \to \O_x\) be a strategy for player 1 in the Gruenhage game played at \(x\) defined by \(\sigma(F) = \rho(F)[x]\) for all \(F \in \S(X)\). 
    
    If \(\langle x_n : n \in \omega \rangle\) is a \(\sigma\)-sequence, i.e., \(x_{n +1} \in \sigma\left(\langle x_1, \dots, x_n\rangle\right)\) for all \(n \in \omega\), then \(x_{n +1} \in \ \rho\left(\langle x_1, \dots, x_n\rangle\right)[x]\) and \(x_n \in \rho\left(\langle x_1, \dots, x_{n - 1}\rangle\right)[x]\).
    As the elements of \(\D\) are symmetric, we have both \(x \in \ \rho\left(\langle x_1, \dots, x_n\rangle\right)[x_{n + 1}]\) and \(x \in \rho\left(\langle x_1, \dots, x_{n - 1}\rangle\right)[x_n]\), which implies \(x_{n + 1} \in \left(\rho\left(\langle x_1, \dots, x_n\rangle\right) +  \rho\left(\langle x_1, \dots, x_{n - 1}\rangle\right)\right)[x_n]\). Thus \(x_{n + 1} \in \rho\left(\langle x_1, \dots, x_{n - 1}\rangle\right)[x_n]\) as \(\rho\left(\langle x_1, \dots, x_{n - 1}\rangle\right) \supset \rho\left(\langle x_1, \dots, x_n\rangle\right)\) since \(\rho\) is a well defined strategy. Hence \(\langle x_n : n \in \omega \rangle\) is a \(\rho\)-sequence and \(x \in \bigcap_{n \in \omega}\rho\left(\langle x_1, \dots, x_{n - 1}\rangle\right)[x_n]\), which implies that \(\langle x_n \in \omega \: : \: n \in \omega\rangle\) must converge to \(x\), which follows by our assumption that \(X\) is Hausdorff.
\end{proof}

\begin{rem}[Proximal Game on Topological Groups]
    The game is played the same but we note that the uniformities chosen by player 1 are formed as follows. On any given turn player 1 picks \(U \in \F\) which corresponds to \(D_U = \bigcup_{a \in [\omega]^{<\omega}}\left(a\Delta V_U\right)^2\). In terms of uniform covers we might can also talk about \(\U_U = \{(a\Delta V_U):a \in [\omega]^{<\omega}\}\). The game then proceeds as usual.
\end{rem}


\section{Musings}
The point of this section is to keep track of thoughts relating to problems whose solutions are maybe not be clear or known (yet). 



%Let \(\F\) be a FUF filter and let \(\tau_{\F}\) be the corresponding topological group.
\subsection{Behaviour \dots}

\subsubsection*{\dots of Sequences}

\begin{itemize}
    \item \(x_n \to x, \in \tau_{\F} \iff a\Delta x_n \to a\Delta x\)
    \item WLOG we can assume \(x = \nothing\)
    \item \(x_n \to x \iff \{x_n\}\) is a network for \(\F\).
\end{itemize}
\begin{prop}
    Let \(\F\) be cofinite filter on \(\omega\) and let \(\langle x_n :n < \omega\} \subset [\omega]^{<\omega}\) such that there exists \(k \in \omega\) such that 
\end{prop}
\begin{proof}
    
\end{proof}
\subsubsection*{\dots of Stars}

Let \(p \in [\omega]^{<\omega}\), let \(F \in \F\). What does \(S :=\st(p, V_F)\) look like?

If \(p \subset F\), then
\begin{itemize}
    \item \( p \in a\Delta V_F \iff p \not \subset a\)
    \item \([\omega\setminus p]^{<\omega} \subset S\) (take \(x \in [\omega\setminus p]^{<\omega}\) and let \(a = x\cap F^C, b = x\cap F\), then \(x = a\Delta b\) and \(b \in V_F\) so \(x \in S\)).
\end{itemize}
If \(p \not\subset F\), then
\begin{itemize}
    \item \( p \in a\Delta V_F \iff p \subset a\)
    \item \(S = [\omega]^{<\omega}\) (\(p\Delta V_F = [F\cup p]^{<\omega} = V_{F\cup p} \in S\), so for any \(x \in [\omega]^{<\omega}\) letting \(a = x\cap (V_{F\cup a})^C\) then \(x = a\Delta F_{F\cup a} \in S\))
\end{itemize}
Actually for any \(F \in \F\) and any \(p \in [\omega]^{<\omega}\), \(\st(p, V_F) = [\omega]^{\omega}\): let \(\alpha = p \cap F^C\), then \(p \in \alpha\Delta V_F =[F\cup\alpha]^{<\omega} = V_{F\cup\alpha} \in \st(p, V_F)\). For \(x\in [\omega]^{<\omega}\), take \(a = x \cap (F\cup\alpha)^C\) so that \(x \in a\Delta V_{F\cup\alpha}\).

\begin{rem}
    The above shows that the only way for player 1 to wins in the proximal game on \(\tau_{\F}\) is for the sequence of points picked by player 2 to converges, i.e., the intersection of the stars will just always be the entire space. 
\end{rem}

\subsubsection*{\dots of Intersections}

Suppose \(\langle a_n:n < \omega\rangle \subset  [\omega]^{<\omega}\) and \(\langle F_n : n < \omega\rangle \subset \F\) and \(b \in \bigcap_{n < \omega} a_n \Delta V_{F_n}\). 

\begin{itemize}
    \item this does not imply that \(a_n \to b\). For example \(b \not \subset a_n\) and \(b \subset F_n\) for all \(n \in \omega\) is possible but we still have \(b \in a_n \Delta V_{F_n}\) for all \(n\).
    \item On the other hand, if \(b \in \bigcap_{n < \omega} a_n \Delta V_{F_n}\) and we want \(a_n \to b\), this would only be possible if \(b \not \subset F_n\) for all \(n > k\) for some \(k\).
\end{itemize}

\subsubsection*{\dots of Filters}
The choice of \(\F\) directly determines the topological properties of \(\tau_{\F}\). Thus if \(\F\) is a countably generated filter then \(\tau_{\F}\) is first countable and it follows that \(\tau_{\F}\) is Proximal. \textcolor{red}{TODO: describe the winning strategy for player 1, it has to exist.}

%%%%%%%%%%%%%%%%%%%%%%%%%%%%%%%%%%%%%%%%%%%%%%%%%%%%%%%%%%%%%%%%%%%%%%%%%%%%%%%%%%%%%%%%%%%%%%%%%%%%%%%%%
%\newpage
%\bibliographystyle{plain}
%\bibliography{bibliography}{}

\end{document}